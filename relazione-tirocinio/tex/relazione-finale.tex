\documentclass[a4paper,10pt]{article}
\usepackage[utf8]{inputenc}

\title{}
\author{}

\textwidth=450pt\oddsidemargin=0pt

\begin{document}

\begin{titlepage}

\begin{center}
{\Large{\textsc{Relazione finale di Tirocinio}}}

\vspace{3mm}
{\small{\bf Corso di Laurea in Informatica }}
\end{center}

\vspace{50mm}

\begin{center}
{\Large{\bf OGGETTO: Studio sull'efficienza di tool}}\\
\vspace{2mm}
{\Large{\bf per l'analisi statica degli smart contract in Ethereum}}
\end{center}

\vspace{3mm}

\begin{center}
{\large{Svolto presso: Alma Mater Studiorum - Università di Bologna}}
\end{center}

\vspace{70mm}

\par
\noindent
\begin{minipage}[t]{0.47\textwidth}
{\large{\bf Svolto da:\\
Melania Ghelli\\
matr. 766608}}
\end{minipage}
\hfill
\begin{minipage}[t]{0.47\textwidth}\raggedleft
{\large{\bf Tutor accademico:\\
Prof. Ugo Dal Lago}}
\end{minipage}

\end{titlepage}

\section{Struttura ospitante e contesto del tirocinio}
Il tirocinio è stato svolto presso l'Università di Bologna.\newline 
Dopo aver concordato con il docente Dal Lago l'argomento della tesi, abbiamo deciso di includere un'attività pratica nel progetto. Il lavoro svolto durante il tirocinio dunque è stato pensato come una base per la redazione del mio elaborato finale.

\section{Attività svolta e capacità acquisite}
L'obiettivo principale di questo tirocinio era quello di confrontare dei software per l'analisi statica.\newline
Il progetto di tesi prevede uno studio delle attuali possibilità di analisi degli smart contract sulla piattaforma di Ethereum. Nello specifico ci siamo focalizzati sull'analisi mirata alla stima dei consumi di questi programmi in termini di gas.\newline
In questo contesto, l'attività di tirocino ha previsto in gran parte una ricerca di software attualmente disponibili che offrano la possibilità di condurre questo specifico tipo di analisi. Il lavoro di ricerca è stato condotto principalmente attraverso la lettura di articoli scientifici, per avere un quadro generale delle ricerche svolte su questo tema.\newline 
Una volta individuato un certo numero di programmi volti ad analizzare smart contract, li ho confrontati fra di loro per capire quanti e quali di questi offrissero la possibilità di dare un bound esplicito ai consumi di gas. Durante questa fase si è reso necessario, talvolta, interfacciarsi direttamente con gli sviluppatori. Questo mi ha dato la possibilità di capire meglio il funzionamento dei loro programmi, ma soprattutto di verificare se fra le funzionalità di questi ci fosse l'analisi del gas.\newline
L'attività finale ha previsto alcuni test, che sono stati condotti sullo stesso insieme di smart contract utilizzando tool differenti. Questo mi ha permesso di ottenere dei risultati utili, che confrontati fra di loro mi permetteranno di fare alcune considerazioni sulla qualità dell'analisi di gas che è possibile fare.\newline
\newline
Capacità acquisite ai fini della formazione personale?

\section{Software utilizzati}
Di seguito un breve riepilogo dei software con cui mi sono interfacciata durante l'attività di ricerca.\newline
Sono tutti programmi pensati per fare analisi statica di smart contract per la piattaforma Ethereum. Molti di questi sono ancora in fase di sperimentazione, dunque non hanno una documentazione esaustiva. Per questa ragione si è reso necessario testarli prima di stabilire quali fossero utili ai fini della ricerca e quali no.\newline
\newline
\begin{description} %software bocciati                     
  \item[EtherTrust] è un software semantico che permette di analizzare smart contract per la verifica delle proprietà di sicurezza. Questo controllo astrae completamente dal gas, dunque non viene fatta alcuna stima dei suoi consumi;
  \item[MadMax] è uno strumento di analisi statica utilizzato per rilevare vulnerabilità legate al gas all'interno dei programmi. Nel fare questo non compie alcuna operazione di calcolo della complessità, dunque non produce un output dove i consumi vengano quantificati. Riesce soltanto ad individuare e segnalare bug nel codice ;
  \item[EthIR] svolge un analisi del bytecode generato a partire dal sorgente scritto in Solidity. L'analisi condotta dal framework produce in output una rappresentazione semplificata del bytecode, rendendolo più leggibile ma soprattutto più adatto per svolgere ulteriori analisi. In questo senso è stato utile per conoscere GASTAP, un software basato su EthIR che però di occupa di stimare i soli consumi di gas.
\end{description}
\begin{description} %software promossi                     
  \item[GASTAP] framework disponibile tramite un'interfaccia web, permette di dedurre degli upper bound ai consumi di gas per ciascuna funzione che compone lo smart contract preso in input. La precisione con cui conduce quest'analisi è pressoché simile a quella del compilatore di Solidity, solc. In alcuni casi GASTAP riesce ad essere anche più preciso di solc, in quanto riesce a dare upperbound finiti in casi in cui invece l'altro programma fallisce ;
  \item[Compilatore solc] è il compilatore ufficiale di Solidity, il linguaggio di programmazione di alto livello messo a disposizione da Ethereum per implementare gli smart contract. Ha una modalità di esecuzione che fornisce un upper bound ai consumi di gas per ciascuno dei metodi del programma preso in input. In tutti i casi testati produce un output, ma spesso il bound dato è infinito;
  \item[KEVM] produce una semantica per il bytecode della EVM. In questo senso è uno strumento simile ad EthIR. Gli sviluppatori di KEVM però mettono a disposizione un'estensione del tool che permette di analizzare e dunque dare una stima dei consumi di gas.
\end{description}

\section{Risultati ottenuti}

Citare ed allegare la tabella di confronto tra solc e GASTAP

\section{Conclusioni}

L'attività che ho svolto durante questo tirocinio è stata utile sotto diversi punti di vista.\newline
In primo luogo mi ha permesso di conoscere una parte del mondo delle crittovalute. Questo topic ha da sempre suscitato molto interesse in me. L'argomento intorno al quale si è sviluppata la mia ricerca mi ha permesso di interfacciarmi con questa realtà, e di conoscere le tecnologie che entrano in gioco in questo sistema, sebbene da un punto di vista circoscritto. Ho acquisito una  conoscenza migliore della blockchain, delle sue possibili applicazioni e del potenziale innovativo che porta con sè. Adesso conosco meglio le monete alternative al bitcoin ed il loro meccanismo di utilizzo.\newline
Un altro aspetto positivo di questo lavoro è stata la ricerca - conoscenza comunità scientifica - lettura di articoli di ricerca.
Infine l'utilizzo di software sperimentali - lo studio dei programmi solidity - ecc

\vspace{30mm}

\par
\noindent
\begin{minipage}[t]{0.47\textwidth}
{\large{Data}}
\end{minipage}
\hfill
\begin{minipage}[t]{0.47\textwidth}\raggedleft
{\large{Firma tirocinante\newline
\newline
\newline
Firma tutor\newline}}
\end{minipage}

\end{document}
