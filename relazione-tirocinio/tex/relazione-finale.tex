\documentclass[a4paper,10pt]{article}
\usepackage[utf8]{inputenc}

\title{}
\author{}

\textwidth=450pt\oddsidemargin=0pt

\begin{document}

\begin{titlepage}

\begin{center}
{\Large{\textsc{Relazione finale di Tirocinio}}}

\vspace{3mm}
{\small{\bf Corso di Laurea in Informatica }}
\end{center}

\vspace{50mm}

\begin{center}
{\Large{\bf OGGETTO: Studio sull'efficienza di tool}}\\
\vspace{2mm}
{\Large{\bf per l'analisi statica degli smart contract in Ethereum}}
\end{center}

\vspace{3mm}

\begin{center}
{\large{Svolto presso: Alma Mater Studiorum - Università di Bologna}}
\end{center}

\vspace{70mm}

\par
\noindent
\begin{minipage}[t]{0.47\textwidth}
{\large{\bf Svolto da:\\
Melania Ghelli\\
matr. 766608}}
\end{minipage}
\hfill
\begin{minipage}[t]{0.47\textwidth}\raggedleft
{\large{\bf Tutor accademico:\\
Prof. Ugo Dal Lago}}
\end{minipage}

\end{titlepage}

\section{Struttura ospitante e contesto del tirocinio}
Il tirocinio è stato svolto presso l'Università di Bologna.\newline 
Dopo aver concordato con il docente Dal Lago l'argomento della tesi, abbiamo deciso di includere un'attività pratica nel progetto. Il lavoro svolto durante il tirocinio dunque è stato pensato come una base per la redazione del mio elaborato finale.

\section{Attività svolta e capacità acquisite}
L'obiettivo principale di questo tirocinio era quello di confrontare dei software per l'analisi statica.\newline
Il progetto di tesi prevede uno studio delle attuali possibilità di analisi degli smart contract sulla piattaforma di Ethereum. Nello specifico ci siamo focalizzati sull'analisi mirata alla stima dei consumi di questi programmi in termini di gas.\newline
In questo contesto, l'attività di tirocino ha previsto in gran parte una ricerca di software attualmente disponibili che offrano la possibilità di condurre questo specifico tipo di analisi. Il lavoro di ricerca è stato condotto principalmente attraverso la lettura di articoli scientifici, per avere un quadro generale delle ricerche svolte su questo tema.\newline 
Una volta individuato un certo numero di programmi volti ad analizzare smart contract, li ho confrontati fra di loro per capire quanti e quali di questi offrissero la possibilità di dare un bound esplicito ai consumi di gas. Durante questa fase si è reso necessario, talvolta, interfacciarsi direttamente con gli sviluppatori. Questo mi ha dato la possibilità di capire meglio il funzionamento dei loro programmi, ma soprattutto di verificare se fra le funzionalità di questi ci fosse l'analisi del gas.\newline
L'attività finale ha previsto alcuni test, che sono stati condotti sullo stesso insieme di smart contract utilizzando tool differenti. Questo mi ha permesso di ottenere dei risultati utili, che confrontati fra di loro mi permetteranno di fare alcune considerazioni sulla qualità dell'analisi di gas che è possibile fare.\newline
\newline
Capacità acquisite ai fini della formazione personale?

\section{Software utilizzati}
Di seguito un breve riepilogo dei software con cui mi sono interfacciata durante l'attività di ricerca.\newline
Sono tutti programmi pensati per fare analisi statica di smart contract per la piattaforma Ethereum. Molti di questi sono ancora in via di sperimentazione, dunque non hanno una documentazione esaustiva. Per questa ragione si è reso necessario testarli prima di stabilire quali fossero utili ai fini della ricerca e quali no.\newline
\newline
Fare elenco dei software.

\section{Risultati ottenuti}

Citare ed allegare la tabella di confronto tra solc e GASTAP

\section{Conclusioni}

Giudizio/opinione personale sul tirocinio svolto

\vspace{30mm}

\par
\noindent
\begin{minipage}[t]{0.47\textwidth}
{\large{Data}}
\end{minipage}
\hfill
\begin{minipage}[t]{0.47\textwidth}\raggedleft
{\large{Firma tirocinante\newline
\newline
\newline
Firma tutor\newline}}
\end{minipage}

\end{document}
