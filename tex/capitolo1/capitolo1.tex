
Lo scopo di questo capitolo è quello di fornire una panoramica dei concetti chiave intorno ai quali si sviluppa l'elaborato.

\section{Blockchain}

Il termine Blockchain - in italiano ``catena di blocchi'' - identifica un registro distribuito e sicuro. In questo senso si può pensare alla blockchain come ad una struttura di dati simile ad una lista crescente, dove le informazioni sono raggruppate in blocchi collegati fra loro.\newline
Ciascun blocco codifica una sequenza di transazioni individuale, e viene concatenato a quello precedente seguendo un ordine cronologico. La concatenazione è irreversibile: ciascun nuovo blocco contiene la firma digitale di quello precedente. In questo modo, modificare un blocco implicherebbe l'invalidazione di tutta la catena successiva.\newline  
La peculiarità di questa struttura risiede nel fatto che sia condivisa: ogni nodo che compone la rete mantiene una copia del registro aggiornata. Per poter aggiungere un blocco è dunque necessario validare l'intera catena, ed ottenere un consenso da parte degli altri nodi della rete. Una volta ottenuto, il nuovo blocco viene trasmesso agli altri componenti in modo tale da aggiornare lo stato della blockchain.\newline
Il processo di validazione dei nuovi blocchi viene realizzato dai miner.
Il loro compito è quello di verificare le transazioni proposte e fare in modo che il nuovo blocco venga linkato alla blockchain. Per fare questo i miner sono chiamati a risolvere un algoritmo proof-of-work, un puzzle crittografico che richiede un significativo costo computazionale per essere risolto.\newline
Questo sistema permette di raggiungere il consenso senza la necessità di un'autorità centrale che faccia da garante. \'E il concetto chiave delle tecnologie basate su blockchain: la possibilità di implementare servizi sicuri senza appoggiarsi a banche, istituzioni pubbliche, ecc.\newline
\newline
Questa nuova tecnologia può essere integrata in diverse aree \cite{K1}, sebbene ad oggi il suo uso più conosciuto sia quello nei sistemi di pagamento che impiegano crittovalute.
Il dato non è poi così sorprendente: la prima blockchain nasce grazie a Satoshi Nakamoto assieme al Bitcoin \cite{K2}. In questo senso il Bitcoin è una
piattaforma di pagamenti, dove la catena di blocchi funge da storico di tutte le transazioni avvenute: una sorta di conto corrente condiviso.\newline

\section{Ethereum, gli Smart Contract e la EVM}

All'interno di quest'elaborato verrà presa in considerazione solo il network Ethereum, una piattaforma decentralizzata basata su una blockchain, che come Bitcoin possiede una propria valuta: l'\textit{ether}.\newline 
Diversamente da quanto vale per le altre crittovalute, Ethereum non è solo un network per lo scambio di moneta, ma un framework che permette l'esecuzione di programmi. Tali programmi prendono il nome di \textit{smart contract}, cioè ``contratti intelligenti''. Sebbene il nome possa suggerire una funzione ben precisa, questi programmi sono usati per computazioni general-purpose, permettendo quindi di realizzare un vasto numero di operazioni.\newline
Gli smart contract sono scritti in linguaggi ad alto livello; fra i vari (Serpent, Viper e LLL) quello più diffuso ad oggi è Solidity \cite{K3}. 
Tale linguaggio object-oriented è pensato solo per lo sviluppo di smart contract che, per poter girare nella rete, vengono poi tradotti in bytecode. Ciascun nodo di Ethereum infatti esegue localmente la Ethereum Virtual Machine, anche detta EVM, una macchina a stack in grado di eseguire un linguaggio di basso livello, ossia bytecode. Questo linguaggio è non tipato, e composto da un piccolo insieme di istruzioni.\newline 

\section{Il ruolo del gas}

Per \textit{gas} si intende l'unità di misura dello sforzo computazionale richiesto dalla EVM per eseguire ciascuna istruzione. Diremo quindi che eseguire uno smart contract costa una certa quantità di gas.\newline
Nello specifico ciascuna istruzione di basso livello ha associato un costo fisso in gas. Per calcolare quindi il consumo totale di un programma Solidity è necessario comprendere in quali istruzioni di basso livello verrà tradotto.\newline
I costi di alcune delle istruzioni EVM \cite{K4} sono riportati nella tabella \ref{tab:gas-costs}.
%(richiamo il nome della tabella utilizzando la label che ho messo sotto)

\begin{table}[h]                        %ambiente tabella
                                        %(serve per avere la legenda)
\begin{center}  %centra nella pagina la tabella

\begin{tabular}{p{5cm}rp{6cm}}  

\hline \hline   %inserisce due righe orizzontali
Istruzione & Costo & Descrizione\\   %& separa le colonne
\hline  %inserisce una riga orizzontale
\bf JUMPDEST & 1 & Indica la destinazione di un'istruzione JUMP\\
\bf POP & 2 & Rimuove un elemento dallo stack\\
\bf PUSHn & 3 & Inserisce un elemento di n byte nello stack\\
\bf ADD, SUB & 3 & Operatori aritmetici\\
\bf AND, OR, NOT, XOR, ISZERO, BYTE & 3 & Operatori logici\\
\bf MUL, DIV & 5 & Operatori aritmetici\\
\bf JUMP & 8 & Salto semplice senza condizione\\
\bf JUMP1 & 10 & Salto condizionale\\
\hline
\bf CALL & 700 & Chiama una transazione\\
\bf CALLVALUE & 9000 & Pagato per un argomento diverso da 0 dell'istruzione CALL\\
\bf SSTORE & 20000 & Salva una parola in memoria. Si paga quando il valore precedente è uguale a 0\\
\hline \hline
\end{tabular}

\caption[legenda elenco tabelle]{Costi espressi in gas di alcune istruzioni della EVM}\label{tab:gas-costs}
\end{center}
\end{table}

Il gas viene pagato in ether dagli utenti che intendono far eseguire una \textit{transazione}. Con transazione si intende l'azione di creare uno smart contract o di chiamarne delle funzioni e di pagare un miner per far eseguire la transazione stessa.\newline
Per fare in modo che la sua transazione venga scelta, un utente stabilisce la quantità di gas che è disposto a pagare per farla portare a termine. Il miner poi, in base a questo parametro, sceglie quali transazioni eseguire. \'E importante fornire un'adeguata somma: i miner, al termine della transazione, possono beneficiare dell'ether destinato all'acquisto di gas che è rimasto inutilizzato. Questo significa che gli smart contract con più probabilità di essere scelti sono quelli degli utenti disposti a pagare di più.\newline
La scelta adottata da Ethereum di far pagare i propri utenti non è triviale. Prima di tutto impedisce loro di sovraccaricare i miner di lavoro sfruttando il potere computazionale della rete. Inoltre scoraggia gli utenti a impiegare troppa memoria, una risorsa preziosa nelle tecnologie basate su blockchain. Infine limita il numero di computazioni eseguite dalla stessa transazione. Questo difende il network intero da attacchi malevoli come i DDoS: la finitezza delle transazioni fa sì che non si possa, ad esempio, far ciclare un programma infinitamente. Per poter disabilitare la rete anche solo per pochi minuti gli hacker dovrebbero pagare delle ingenti somme.\newline
Dal momento che il gas viene pagato anticipatamente, può succedere che durante la sua esecuzione un programma ecceda la quantità che ha a disposizione. Questo comportamento è  indesiderato, in quanto comporta spiacevoli conseguenze. La più immediata è il blocco della transazione: la computazione non giunge a termine, l'utente non ottiene il risultato desiderato e l'ether pagato per il gas va perso.\newline 
La conseguenza indiretta invece è il possibile blocco permanente dello smart contract. Quando durante l'esecuzione di una transazione un'istruzione richiede una quantità di gas superiore a quella disponibile, la EVM solleva un'eccezione di tipo \textit{out-of-gas} e interrompe la transazione. Qualora lo smart contract non preveda una gestione di questo tipo di eccezione resterà bloccato per sempre.\newline



