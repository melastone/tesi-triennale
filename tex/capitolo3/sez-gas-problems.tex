\section{Stimare i Consumi di gas}


	\textbf{Perchè stimare i consumi di gas è importante?}\newline
    \newline
	\indent I consumi di gas sono associati alle operazioni di basso livello, 
	dunque specificati solo per il bytecode EVM \cite{wood2014ethereum}. Dal momento che però gli
	smart contract sono sviluppati utilizzando linguaggi di alto livello (come
	ad es. Solidity \cite{ethereum/solidity_2019}), è difficile per lo sviluppatore conoscere i 
	costi del proprio programma durante la fase di sviluppo. Per di più la traduzione 
	dei costrutti di alto livello in bytecode fa sì che stimare i constumi tramite l'analisi 
	statica sia una sfida non triviale. L'impiego dell'analisi statica si rende indispensabile, in quanto il costo totale in gas richiesto per eseguire un programma dipende anche da altri fattori.\newline
	\indent Semplificando potremmo dire che i costi totali dipendono da:
	\begin{enumerate}
	\item il costo intrinseco di ciascuna istruzione di basso livello; questi valori sono fissati (vedi Tabella \ref{tab:gas-costs}).
	\item i costi determinati dalla creazione di un contratto o dalla chiamata di un altro programma. Questi sono determinati dalle istruzioni \texttt{CREATE} , \texttt{CALL} and \texttt{CALLCODE}.
	\item eventuali costi aggiunti, che vengono addebitati nel caso in cui la memoria richiesta dal programma superi una certa soglia.
	\end{enumerate}

	Mentre alcuni di questi valori possono essere facilmente stimati, altri possono essere determinati soltanto durante l'esecuzione del contratto. Un modo per sopperire a questa difficoltà è l'analisi statica: soltanto delle tecniche precise ci permettono di stimare questi valori, poichè consentono di calcolare in anticipo quali ``sorprese'' riserverà il codice durante la sua esecuzione.\newline
	\indent Dal momento che il gas viene pagato anticipatamente, può succedere che durante la sua
	esecuzione un programma ecceda la quantità che ha a disposizione. Come conseguenza, la EVM
	sollleva un'eccezione di tipo \textit{out-of-gas} e abortisce la transazione. Un contratto
	che non gestisce bene l'eventuale interruzione di una transazione è soggetto ad una 
	vulnerabilità legata al gas. Una panoramica su questo tipo di vulnerabilità viene fornita dagli autori di \textsc{MadMax} ~\cite{grech2018madmax}. Generalmente questi programmi, che vengono etichettati
	come rischiosi, saranno bloccati in modo permanente.
	Quando si eccede il gas disponibile un'altra conseguenza più immediata è il
	blocco della transazione: la computazione non giunge a termine, l'utente non ottiene il
	risultato desiderato e l'ether pagato per il gas va perso.\newline
	\indent Un altro limite all'esecuzione dei contratti è dovuto al protocollo adottato da Ethereum.
	Questo infatti pone un limite superiore alla quantità di gas che ciascun blocco può consumare. Dal momento che le transazioni vengono raggruppate in blocchi,
	tale limite influenza anche queste: il costo di ciascuna transazione non può superare il limite superiore del blocco al quale appartiene. Può succedere infatti che
	se l'esecuzione di una certa funzione aumenta nel tempo, ad un certo punto non sia
	più possibile portarla avanti a causa del superamento del limite massimo di gas.
	Conoscere a priori i consumi può aiutare anche ad evitare questo tipo di errori.\newline
	\indent Va inoltre considerato che se l'utente della rete ha modo di conoscere il costo
	di una computazione nel suo caso pessimo, ha anche modo di confrontare fra di loro dei programmi
	semanticamente equivalenti, al fine di prediligere quello che consuma meno.  In questo 
	senso la stima dei costi di gas costituirebbe l'uinica fonte di risparmio: considerato
	che le transazioni sotto una certa soglia di \textit{gasPrice} rischiano di non essere 
	accettate dai miner, i committenti non hanno margine di risparmio.\newline
	\indent Una stima affidabile del gas aiuta un utente a stabilire un prezzo per ciascuna
	unità di gas in linea con l'utilità della sua transazione. Infatti una quantità 
	insufficiente per completare la transazione comporta la perdita dei soldi investiti,
	senza che la transazione venga eseguita. Al tempo stesso una sovrastima fa sì che i
	miner assumano un atteggiamento diffidente, abbassando la probabilità che la stessa
	venga scelta.\newline
	\indent Conoscere un limite ai consumi di gas del proprio programma assicura all'utente
	che se la quantità di gas investito supera il bound l'esecuzione verrà portata a termine
	 enza incorrere in spiacevoli sorprese.

	%Si potrebbe aggiungere anche il fatto delle callbak. Quando si chiama un metodo di un altro contratto,e quindi non si conosce lo stato dell'ambiente, poter calcolare i consumi ci permette di mettere a disp una qta di gas adeguata ed eseguire la callback con successo!

	\newpage
