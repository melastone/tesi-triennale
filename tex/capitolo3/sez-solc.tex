    
\section{Compilatore solc}

Si tratta del compilatore ufficiale del linguaggio Solidity, utilizzabile da linea di comando.\newline
\indent Il comando \verb|solc --help| fornisce la spiegazione di ciascuna delle opzioni con cui può essere lanciato.

\lstset{
    style=cmd-line,
    literate={~} {$\sim$}{1}
}

\begin{lstlisting}
$~solc --help

solc, the Solidity commandline compiler.

This program comes with ABSOLUTELY NO WARRANTY. This is free software, and you
are welcome to redistribute it under certain conditions. See 'solc --license'
for details.

Usage: solc [options] [input_file...]

...

Allowed options:
  --help               Show help message and exit.
  --version            Show version and exit.
  --license            Show licensing information and exit.
    
...

  --gas                Print an estimate of the  
                       maximal gas usage for each 
                       function. 
 
\end{lstlisting}



Per condurre i nostri test abbiamo utilizzato il compilatore con l'opzione \verb|--gas|. In questa modalità il compilatore è in grado di determinare soltato dei bound costanti; in tutti gli altri casi produce \verb|infinite| come valore di output.\newline 
\indent Ecco un esempio del suo utilizzo sul programma raffigurato in figura \ref{fig:gstp-example}.

\begin{minipage}{\linewidth}
\begin{lstlisting}
$~solc --gas example.sol 

======= example.sol:Example =======
Gas estimation:
construction:
   5093 + 32800 = 37893
external:
   set(uint256):	20205

\end{lstlisting}
\end{minipage}


Confrontando i risultati ottenuti con quelli prodotti da GASTAP si evince che la stima del gas consumato dalla funzione \verb|set()| corrisponde a quela determinata da solc. Dunque l'uso dell'analisi statica non fa sì che si perda accuratezza nel calcolo.\newline
\indent Nella sezione 4.3.2 l'esempio verrà ripreso al fine di comprendere il bound.\newline
