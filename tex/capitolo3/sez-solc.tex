    
\section{Compilatore solc}

Si tratta del compilatore ufficiale del linguaggio Solidity, utilizzabile da linea di comando. Il comando \verb|solc --help| fornisce la spiegazione di ciascuna delle opzioni con cui può essere lanciato.\newline
\indent Per condurre i nostri test abbiamo utilizzato il compilatore con l'opzione \verb|--gas|. In questa modalità solc è in grado di determinare soltato dei bound costanti; in tutti gli altri casi produce \verb|infinite| come valore di output. Questo comportamento resta tutt'oggi una forte limitazione nell'utilizzo del compilatore, come provano le discussioni presenti nelle community degli utenti di Ethereum ~\cite{infinite-gas-github-issue, infinite-gas-stack-exchange, gas-costs-stack-exchange}. In molti casi questo limite viene attribuito ad una cattiva gestione di solc dei \texttt{JUMP} all'indietro nel bytecode EVM.\newline
\indent Ecco un esempio dell'utilizzo del compilatore sul programma raffigurato in Figura \ref{fig:gstp-example}.

\begin{center}
 \begin{minipage}{\linewidth}
\begin{lstlisting}
$~solc --gas example.sol 

======= example.sol:Example =======
Gas estimation:
construction:
   5093 + 32800 = 37893
external:
   set(uint256):	20205

\end{lstlisting}
\end{minipage}
\end{center}

Confrontando i risultati ottenuti con quelli prodotti da GASTAP si evince che la stima del gas consumato dalla funzione \verb|set()| corrisponde a quella determinata da solc. Dunque l'uso dell'analisi statica non fa sì che si perda accuratezza nel calcolo.\newline
\indent Nella Sezione 4.3.2 l'esempio verrà ripreso al fine di comprendere il bound.\newline
