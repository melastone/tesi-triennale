% !TeX spellcheck = it                                        

\documentclass[12pt,a4paper,openright,twoside]{report}
%openright: apre i capitoli a destra
%twoside: serve per fare un documento fronteretro


\usepackage[italian]{babel}
%\usepackage[latin1]{inputenc}
\usepackage[utf8]{inputenc}
\usepackage[T1]{fontenc} %aumenta il tempo di compilazione


%libreria per impostare il documento (headers & footers)
\usepackage{fancyhdr}


%libreria per avere l'indentazione all'inizio dei capitoli
\usepackage{indentfirst}


%%%%%%%%%libreria per mostrare le etichette
%\usepackage{showkeys}


%%%%%%%%%%%%%%%%%%%%%%%%%%%%%%%%%%%%%%%%%libreria per inserire grafici
%\usepackage{graphicx}


%%%%%%%%%%%%%%%%%%%%%%%%%%%%%%%%%%%%%%%%%libreria per utilizzare font particolari ad esempio  \textsc{}
%\usepackage{newlfont}


%%%%%%%%%%%%%%%%%%%%%%%%%%%%%%%%%%%%%%%%%librerie matematiche
%\usepackage{amssymb}
%\usepackage{amsmath}
%\usepackage{latexsym}
%\usepackage{amsthm}


\oddsidemargin=30pt \evensidemargin=20pt%impostano i margini
\hyphenation{sil-la-ba-zio-ne pa-ren-te-si}%serve per la sillabazione: tra parentesi vanno inserite come nell'esempio le parole che latex non riesce a tagliare nel modo giusto andando a capo.


%%%%%%%%%%%%%%%%%%%%%%%%%%%%%%%%%%%%%%%%%comandi per l'impostazione della pagina, vedi il manuale della libreria fancyhdr per ulteriori delucidazioni
\pagestyle{fancy}\addtolength{\headwidth}{60pt}
\setlength{\headheight}{52pt}
\renewcommand{\chaptermark}[1]{\markboth{\thechapter.\ #1}{}}
\renewcommand{\sectionmark}[1]{\markright{\thesection \ #1}{}}
\rhead[\fancyplain{}{\bfseries\leftmark}]{\fancyplain{}{\bfseries\thepage}}
\cfoot{}


\linespread{1.3}                        %comando per impostare l'interlinea


%%%%%%%%%%%%%%%%%%%%%%%%%%%%%%%%%%%%%%%%%definisce nuovi comandi




\textwidth=450pt\oddsidemargin=0pt
\begin{document}

%scrivo il frontespizio
\begin{titlepage}
\begin{center}

{{\Large{\textsc{Alma Mater Studiorum $\cdot$ Universit\`a di
Bologna }}}}

%inserisce le 2 righe orizzontali
\rule[0.1cm]{15.8cm}{0.1mm}
\rule[0.5cm]{15.8cm}{0.6mm}

{\small{\bf SCUOLA DI SCIENZE\\
Corso di Laurea in Informatica }}
\end{center}
\vspace{15mm}
\begin{center}
{\LARGE{\bf Stimare i consumi di gas}}\\
\vspace{3mm}
{\LARGE{\bf nell'esecuzione di Smart Contract}}\\
\vspace{3mm}
{\LARGE{\bf in Ethereum}}\\
\end{center}
\vspace{40mm}

\par
\noindent
\begin{minipage}[t]{0.47\textwidth}
{\large{\bf Relatore:\\
Chiar.mo Prof.\\
Ugo Dal Lago}}
\end{minipage}
\hfill
\begin{minipage}[t]{0.47\textwidth}\raggedleft
{\large{\bf Presentata da:\\
Melania Ghelli}}
\end{minipage}
\vspace{20mm}
\begin{center}
{\large{\bf Sessione II\\%inserire il numero della sessione in cui ci si laurea
Anno Accademico 2018-2019 }}%inserire l'anno accademico a cui si � iscritti
\end{center}
\clearpage{\pagestyle{empty}\cleardoublepage}%non numera l'ultima pagina sinistra
\end{titlepage}

%\begin{titlepage}                       %crea un ambiente libero da vincoli
                                        %   di margini e grandezza caratteri:
                                        %   si pu\`o modificare quello che si
                                        %   vuole, tanto fuori da questo
                                        %   ambiente tutto viene ristabilito
%
%\thispagestyle{empty}                   %elimina il numero della pagina
%\topmargin=6.5cm                        %imposta il margina superiore a 6.5cm
%\raggedleft                             %incolonna la scrittura a destra
%\large                                  %aumenta la grandezza del carattere
                                        %   a 14pt
%\em                                     %emfatizza (corsivo) il carattere
%Questa \`e la \textsc{Dedica}:\\
%ognuno pu\`o scrivere quello che vuole, \\
%anche nulla \ldots                      %\ldots lascia tre puntini
%\newpage                                %va in una pagina nuova
%
%%%%%%%%%%%%%%%%%%%%%%%%%%%%%%%%%%%%%%%%
%\clearpage{\pagestyle{empty}\cleardoublepage}%non numera l'ultima pagina sinistra
%\end{titlepage}

\pagenumbering{roman}                   %serve per mettere i numeri romani

\tableofcontents                        %crea l'indice
%%%%%%%%%%%%%%%%%%%%%%%%%%%%%%%%%%%%%%%%%imposta l'intestazione di pagina
\rhead[\fancyplain{}{\bfseries\leftmark}]{\fancyplain{}{\bfseries\thepage}}
\lhead[\fancyplain{}{\bfseries\thepage}]{\fancyplain{}{\bfseries
INDICE}}

%%%%%%%%%%%%%%%%%%%%%%%%%%%%%%%%%%%%%%%%%non numera l'ultima pagina sinistra
%\clearpage{\pagestyle{empty}\cleardoublepage}



\chapter{Introduzione}            
\pagenumbering{arabic}

%%%%%%%%%%%%%%%%%%%%%%%%%%%%%%%%%%%%%%%%%imposta l'intestazione di pagina
\rhead[\fancyplain{}{\bfseries
INTRODUZIONE}]{\fancyplain{}{\bfseries\thepage}}
\lhead[\fancyplain{}{\bfseries\thepage}]{\fancyplain{}{\bfseries
INTRODUZIONE}}

%%%%%%%%%%%%%%%%%%%%%%%%%%%%%%%%%%%%%%%%%aggiunge la voce        Introduzione nell'indice
%\addcontentsline{toc}{chapter}{Introduzione}


Le innovazioni tecnologiche introdotte negli ultimi decenni hanno rivoluzionato la nostra società. Il risultato che ne deriva è che numerosi settori stanno cambiando, muovendosi verso una realtà sempre più digitale. Se da una parte questo ha costituito un progresso, dall'altra ha creato nuove opportunità per i cybercriminali.\newline
Oggi il principale obiettivo della sicurezza informatica è proprio quello di trovare soluzioni adattabili alle nuove infrastrutture, come i sistemi IoT che si stanno diffondendo sempre di più. Con l'impiego di queste nuove tecnologie nell'industria i punti di accesso alla rete aziendale sono aumentati, moltiplicando la tipologia ed il numero di minacce. In questo contesto il rischio che si corre è maggiore, poichè legato alla violazione di dati sensibili o addirittura alla compromissione dei processi di produzione. %In questo scenario la cybersecurity è diventata una %disciplina di fondamentale importanza.
\newline
La blockchain nasce in questo contesto, riscuotendo un grande successo grazie al potenziale innovativo che porta con sé. Questa nuova tecnologia permette l'esecuzione di programmi in modo distribuito e sicuro, senza la necessitá di un ente centrale che faccia da garante. Il paradigma
trova applicazioni nei settori piú disparati, offrendo innovazione grazie alla
possibilitá di fare a meno di banche o istituzioni pubbliche.\newline
Ma in cosa consiste dal punto di vista informatico?
Blockchain, catena di blocchi.
Registro distibuito, organizzato in blocchi legati tra loro. Ad interagire con essa sono i miner, coloro che fisicamente realizzano le così dette transazioni. 
Qual è il ruolo del miner nello specifico??\newline
\newline
Grazie alla crittografia (sicurezza informatica comprende una serie di cose, tra cui la crittografia) è stato possibile implementare delle monete virtuali. Da questi studi si è arrivati poi a pensare al bitcoin, che è stato il primo esperimento condotto con successo.\newline
Ciò che indeboliva le crittovalute era il fatto di essere ''centralizzate'', cioè di passare per un ente che ne garantisse l'uso (una sorta di banca).\newline
La blockchain nasce come necessità di superare quest'ostacolo, perché finalmente introduce la possibilità di farne a meno.
Il primo esempio storico di moneta digitale si ha con il Bitcoin. Già da lui si parla di assenza di autorità centrale e rete distibuita - usando algoritmi proof-of-work. Questi permettono di raggiungere un consenso distribuito attraverso tutta la rete. E' il concetto chiave x capire come riusciamo ad evitare di affidarci ad un singolo ente es. Banca.\newline
Tra i sistemi nati grazie alla blockchain troviamo Ethereum, una piattaforma che mette a disposizione un linguaggio di programmazione di alto livello. Questo linguaggio puó essere utilizzato dagli utenti per implementare dei programmi, i così detti smart contract.\newline
In che cosa differisce dal Bitcoin?\newline
\newline
Gli smart contract possono essere eseguiti sulla rete di Ethereum solo al fronte di un pagamento anticipato. Per ragioni di sicurezza a ciascuna istruzione di basso livello è associato un costo monetario. Dunque eseguire un programma costerá tanto quante sono le istruzioni che lo compongono.
Il costo di ciascuna istruzione è espresso in termini di gas, una sorta di carburante che viene pagato in ether, la crittovaluta di Ethereum.
Dal momento che il gas viene pagato in anticipo, potrebbe accadere che l'esecuzione di un programma ecceda la quantitá messa a disposizione. In questi casi la computazione non giunge al termine, risultando nella perdita delle risorse investite dall'utente. Oltre a questo comportamento indesiderato l'esaurimento del gas disponibile puó avere conseguenze pericolose.
Un programma che non gestisce correttamente queste situazioni viene etichettato come vulnerabile. La conseguenza piú diretta è il blocco del contratto, che puó essere anche permanente. La pericolositá peró risiede nel fatto che questo tipo di programmi diventano un bersaglio facile per attacchi malevoli. Vedremo come queste vulnerabilitá possono essere sfruttate per ottenere
comportamenti dannosi per la rete.\newline
Dato il valore monetario associato agli smart contract il rischio che si corre in caso di attacchi informatici è una perdita di denaro. Per questo motivo è necessario individuare possibili criticitá nel codice prima della sua esecuzione. In questo contesto l'analisi statica dei programmi costituisce un potente
strumento di prevenzione.\newline
\newline
All'interno di questo elaborato ci concentreremo solo sulle tecniche di analisi dei consumi di gas. Poter conoscere a priori quest'informazione permetterebbe non solo un investimento adeguato da parte degli utenti, ma anche uno strumento di prevenzione da possibili attacchi.\newline
Attualmente non esistono strumenti in grado di calcolare con precisione la quantitá di gas richiesto durante una computazione. Cercheremo di capirne le ragioni ma soprattutto di individuare dei margini di miglioramento.\newline
\newline


%%%%%%%%%%%%%%%%%%%%%%%%%%%%%%%%%%%%%%%%%non numera l'ultima pagina sinistra
\clearpage{\pagestyle{empty}\cleardoublepage}

\chapter{Background}

\lhead[\fancyplain{}{\bfseries\thepage}]{\fancyplain{}{\bfseries\rightmark}}

L'analisi statica è un processo di valutazione della correttezza dei programmi che rientra tra le tecniche di verifica del software. L'aggettivo \textit{statica} identifica una serie di controlli che possono essere effettuati sul codice prima della sua esecuzione. In questo si differenzia dall'analisi dinamica, una tecnica complementare che comprende quei controlli che vengono invece effettuati a runtime.\newline
\indent L'analisi statica solitamente è il primo controllo che viene effettuato sul codice durante lo sviluppo del programma.
% Perchè l'analisi statica non è common practice?
I vantaggi apportati da questo tipo di controllo del codice sono numerosi. I bug possono essere individuati per tempo, evitando comportamenti inaspettati da parte dei programmi. Inoltre, a seconda del tool utilizzato, è possibile migliorare il codice dal punto di vista della leggibilità, della strutturazione o della performance.\newline
\indent Tuttavia l'analisi statica resta una pratica poco diffusa tra gli sviluppatori \cite{johnson2013don}. Le principali cause sono da ricondursi agli output prodotti dai tool di analisi: l'elevato numero di warning così come la presenza di falsi positivi rendono gli strumenti meno affidabili. Oltre a questo, gran parte degli informatici afferma di non utilizzare questi tool di verifica a causa del sovraccarico di lavoro: spesso i ritmi aziendali così come la scarsa collaborazione dei team di sviluppo fanno sì che non ci siano i presupposti per dedicare spazio sufficiente all'analisi del codice.\newline


\section{Analisi Statica vs. Analisi Dinamica}

Per analisi statica si intende l'analisi dei programmi dal punto di vista del codice che li compone, vale a dire senza doverli eseguire. L'analisi può essere fatta sia sul codice sorgente, che sul codice oggetto, ossia  il prodotto della compilazione.\newline
\indent L'analisi statica viene condotta su tre dimensioni: esaminando la struttura del programma, costruendo un modello che rappresenti i possibili stati del codice e ragionando sul possibile comportamento in fase di esecuzione ~\cite{ernst-ijcai97}. 
Rientrano in questa categoria la verifica formale dei programmi e le ottimizzazioni a tempo di compilazione. L'analisi statica viene spesso implementata da tool automatici, e garantisce proprietà di correttezza. Le principali critiche mosse nei confronti di questa tecnica derivano dal fatto che possa portare a dei \textit{falsi positivi}, cioè situazioni in cui viene segnalata una vulnerabilità nel codice sebbene non sia stata violata alcuna regola.
% Perchè vale questo? 
% - le proprietà sono indecidibili
% - vogliamo la soundness
La ragione di questo comportamento è legata alle proprietà che ci poniamo di verificare: la terminazione di un programma, ad esempio, è di per sè una proprietà indecidibile. \'E dimostrato matematicamente che non esiste alcun metodo di analisi statica che sia al tempo stesso \emph{corretto} e \emph{completo} e che non sia limitato dalle risorse che utilizza (es. memoria, tempo)\footnote{Questo risultato è dato dal Teorema di Rice} ~\cite{ausiello2003linguaggi}. Dunque, per ottenere degli strumenti realmente utili, siamo costretti a rinunciare alla completezza dell'analisi, preferendo dei risultati che siano corretti. Ci accontentiamo di strumenti che producono risultati approssimati, talvolta anche diversi da quelli attesi.\newline
\indent L'analisi dinamica identifica quei controlli che possono essere effettuati sul programma soltanto durante la sua esecuzione, che sia su un processore reale o virtuale. Il software testing rientra in questa categoria. Per condurre questo tipo di controllo è necessario fornire un input ben preciso e analizzare poi il comportamento del programma. Occorre inoltre stabilire a priori \emph{che cosa} si vuole misurare. Sebbene questa analisi sia più veloce rispetto alla prima, non garantisce la stessa correttezza. Per essere rigorosa infatti l'analisi dinamica dovrebbe coprire ogni possibile configurazione del programma.\\


\section{Tecniche di Analisi Statica in Informatica}

Le tecniche di analisi possono essere suddivise in due tipologie in base ai risultati prodotti. La prima categoria comprende i tool di analisi volti a localizzare bug nel codice. La seconda identifica invece un gruppo di software con una forte base logica, che utilizzano tecniche matematiche per la verifica di specifiche prorietà del programma.\newline
\indent Di seguito daremo una panoramica sulle principali tecniche ~\cite{analisi-statica-unina} di analisi statica.
% Espandiamo la sezione
% Differenziamo le tecniche manuali da quelle automatizzabili

    \subsection{La Compilazione}
    
    % Spieghiamo in dettaglio che per essere eseguita la compilazione richiede un po' di analisi

    Tutti i compilatori per eseguire la traduzione del codice dorgente in codice oggetto applicano l'analisi statica. L'operazione di compilazione, intesa come \emph{traduzione automatica}, può essere suddivisa in due macro-fasi: dal codice sorgente alla generazione della forma intermedia, e dalla forma intermedia al codice oggetto, cioè il prodotto finale. La prima fase è quella che fa più utilizzo di analisi statica; il compilatore esegue in sequenza delle trasformazioni sul codice, chiamate analisi lessicale, sintattica e semantica. \`E durante quest'ultimo passaggio che il programma viene sottoposto ai controlli relativi ai vincoli del linguaggio ~\cite{gabbrielli2011linguaggi}: si controllano le dichiarazioni delle variabili, la coerenza dei tipi, il numero dei parametri delle funzioni ecc. 
    In generale questi controlli variano a seconda del linguaggio di programmazione.\newline
    \indent La compilazione dunque non costituisce una vera e propria tecnica, ma piuttosto un tipo di controllo sul codice che non può essere risparmiato.\newline
    
    \subsection{Tecniche Manuali}
    
    Consideriamo \textit{manuali} quelle tecniche che non possono essere automatizzate da un software, ma richiedono l'interazione umana per poter essere realizzate. Di seguito ne citiamo alcune.\newline
    
        \subsubsection{Code Reading}

        Come suggerisce il termine stesso si tratta della rilettura del codice da parte di una persona. Sebbene i bug identificabili possono variare in base a diversi fattori (es. numero di persone, conoscenza del codice, livello di esperienza) questa operazione può portare alla luce difetti che invece il compilatore non rileva. Commenti inconsistenti con il codice, nomi di variabili errati, loop infiniti, codice non strutturato, sono solo alcuni di questi. L'efficacia di questa tecnica è limitata se colui che legge il codice è la stessa persona ad averlo sviluppato.\newline

        \subsubsection{Code Reviews}

        Generalmente adottata in contesti aziendali. Identifica un controllo del codice fatto in gruppo, il quale viene costituito secondo requisiti specifici. \`E una riunione dove lo sviluppatore è chiamato a leggere il codice ad alta voce di fronte ad altri esperti, che possono commentare il programma con lo scopo di individuare gli errori; in questo modo possono essere rilevati dal 30 al 70\% di quelli presenti nel programma.\newline

        \subsubsection{Walktrough}

        Molto simile alla tecnica precedente per le modalità in cui viene effettuata, poiché prevede la riunione di un gruppo di persone. Differisce negli obiettivi: cerca di trovare dei difetti nel comportamento del programma, e per farlo simula l'esecuzione del codice a mano.\newline
        
    \subsection{Tecniche Automatizzabili}
    
    Queste tecniche di revisione del codice possono essere automatizzate, al fine di implementare dei tool di analisi. Ne riportiamo alcuni esempi.\newline

        \subsubsection{Control Flow Analysis}

        Prevede la rappresentazione del codice attraverso un grafo chiamato CFG (\textit{Control Flow Graph}), dove ciascun nodo rappresenta un'istruzione o un predicato, mentre gli archi il passaggio del flusso di controllo.
        Successivamente il grafo viene analizzato, al fine di rilevare anomalie nel programma quali non strutturazione o iraggiungibilità del codice.\newline
        
        \subsubsection{Data Flow Analysis}
        
        La tecnica di analisi del data flow solitamente rientra nella categoria dei controlli dinamici. 
        Analizza l'evoluzione delle variabili durante il tempo di esecuzione, al fine di rilevare anomalie.
        Parte di questi controlli possono essere effettuati anche staticamente, permettendo di rilevare parte dei comportamenti anomali del programma, come l'uso delle variabili prima della loro dichiarazione, o l'annullamento prima dell'utilizzo.\newline
        
        \subsubsection{Esecuzione Simbolica}
        
        Consiste nell'esecuzione del programma con dei valori di input simbolici (es. espressioni) piuttosto che con i valori effettivi. Può risultare molto difficile da realizzare in caso di istruzioni if, poichè rende complesso valutare la condizione. Un altro caso che viene mal gestito è quello dei cicli, sia determinati (nel caso dipendano dal valore di una variabile) che non.\newline

\section{Analisi Statica di Smart Contract}

L'impiego delle tecniche di analisi statica per la verifica degli smart contract non è molto diffuso. Principalmente perchè data la dimensione limitata di questi programmi non si ritiene necessario il suo impiego.\newline
\indent In parte l'impopolarità dell'analisi statica è dovuta anche alla difficile rappresentazione del bytecode EVM. Decompilare le istruzioni di basso livello al fine di ottenere una rappresentazione migliore che funga da base per una buona analisi richiede un notevole sforzo. Un altro fattore a rendere poco appetibile l'applicazione di queste tecniche al mondo degli smart contract è il rischio di ottenere falsi positivi.\newline

\section{Tool per l'Analisi}
Durante questo lavoro è stato preso in considerazione un certo numero di software che implementano tecniche di analisi statica orientata alla verifica degli smart contract. Di seguito ne vedremo alcuni.

\subsection{Verificare le Proprietà di Sicurezza}

I seguenti software sono stati pensati per verificare la sicurezza dei programmi di Ethereum.\newline
\indent Il primo è uno strumento completo, per cui si può etichettare uno smart contract come \emph{sicuro} o meno. Il secondo invece è in grado di indivuare dei comportamenti anomali dei programmi causati soltanto dall'esaurimento del gas. Dunque la sua verifica comprende una tipologia circoscritta di proprietà di sicurezza.\newline

\begin{description}[labelindent=1cm]    %crea un elenco descrittivo

    \item[EtherTrust] ~\cite{grishchenko2018foundations} questo framework offre la possibilità di analizzare i programmi al fine di verificarne le proprietà di sicurezza. Tali proprietà, come ad es. la \textit{single-entrancy}, per poter essere verificate devono prima essere modellate.\newline
    \indent Per condurre la sua analisi EtherTrust produce una rappresentazione astratta del bytecode EVM nella forma di clausole di Horn. Successivamente questa rappresentazione viene data in input ad un SMT solver, il quale verifica che siano rispettate delle proprietà di sicurezza ben precise. EtherTrust garantisce la proprietà di correttezza.\newline

    \item[MadMax] ~\cite{grech2018madmax} attraverso la combinazione di più tecniche di analisi statica (analisi Data Flow e Control Flow) questo software è in grado di verificare smart contract al fine di scoprire bug legati all'esaurimento del gas disponibile.\newline
    \indent MadMax individua una serie di vulnerabilità \textit{gas-focused} in modo da definire dei pattern da ricercare attraverso l'analisi dei programmi. Questa viene condotta a partire da una rappresentazione intermedia (IR) del codice, ottenuta tramite la decompilazione del bytecode EVM.\newline

\end{description}

\subsection{Rappresentare il Bytecode EVM}

I prossimi tool utilizzano tecniche di analisi statica per fornire una miglior rappresentazione del bytecode. I risultati che si ottengono dalla loro esecuzione possono essere utilizzati per un'analisi statica volta alla verifica delle proprietà del codice.\newline

\begin{description}

    \item[KEVM] ~\cite{hildenbrandt2017kevm} produce una semantica formale per la EVM. Gli autori del programma sottolineano che la loro rappresentazione del byteocode si presta facilmente all'applicazione di tecniche di analisi, e forniscono come esempio un tool per stimare i consumi di gas degli smart contract.\newline

    \item[EthIR] ~\cite{albert2018ethir} è un framework di analisi del bytecode di EVM. A partire dalle istruzioni di basso livello, che vengono rappresentate tramite grafi CFG dal tool Oyente ~\cite{melonproject/oyente}, EthIR produce una rappresentazione \textit{Ruled Based} (RB). Tale modellizzazione può essere utilizzata per desumere proprietà del bytecode, applicando delle ulteriori tecniche di analisi statica.\newline
    
\end{description}

\subsection{Stimare i consumi di GAS}

L'ultima categoria di software che vedremo è la più interessante dal punto di vista della ricerca che abbiamo condotto. Si tratta di programmi che tramite la combinazione di tecniche di analisi statica rilevano e forniscono un bound ai consumi di gas dei programmi esaminati.\newline
\indent Li citeremo per completezza, per poi trattarli in modo più dettagliato nei capitoli successivi.\newline

\begin{description}

    \item[solc] ~\cite{solidity-docs} è il compilatore ufficiale di Solidity. Tra le opzioni di utilizzo c'è la modalità \textit{gas}, dove l'output prodotto è una stima della quantità di gas richiesto dal programma. Nella maggior parte dei casi il risultato prodotto è infinito.\newline

    \item[GASTAP] ~\cite{DBLP:journals/corr/abs-1811-10403} è la prima piattaforma sviluppata in grado di analizzare smart contract al fine di dare un upper bound ai consumi di gas dello stesso. Questo software è ancora in via di sviluppo, perciò presenta ancora delle limitazioni. Tuttavia si distingue per la precisione nella stima dei bound, riuscendo a fornire un analisi più precisa rispetto ad altri programmi che implementano le stesse funzionalità. 

\end{description}


%\listoffigures                          %crea l'elenco delle figure
%%%%%%%%%%%%%%%%%%%%%%%%%%%%%%%%%%%%%%%%%non numera l'ultima pagina sinistra
%\clearpage{\pagestyle{empty}\cleardoublepage}
%\listoftables                           %crea l'elenco delle tabelle
%%%%%%%%%%%%%%%%%%%%%%%%%%%%%%%%%%%%%%%%%non numera l'ultima pagina sinistra
%\clearpage{\pagestyle{empty}\cleardoublepage}
%\chapter{Primo Capitolo}                %crea il capitolo
%%%%%%%%%%%%%%%%%%%%%%%%%%%%%%%%%%%%%%%%%imposta l'intestazione di pagina
%\lhead[\fancyplain{}{\bfseries\thepage}]{\fancyplain{}{\bfseries\rightmark}}
%\pagenumbering{arabic}                  %mette i numeri arabi
%Questo \`e il primo capitolo.
%\section{Prima Sezione}                 %crea la sezione
%Questa \`e la prima sezione.

%Ora vediamo un elenco numerato:         %crea un elenco numerato
%\begin{enumerate}
%\item primo oggetto
%\item secondo oggetto
%\item terzo oggetto
%\item quarto oggetto
%\end{enumerate}

%\begin{figure}[h]                       %crea l'ambiente figura; [h] sta
                                        %   per here, cio� la figura va qui
%\begin{center}                          %centra nel mezzo della pagina
                                        %   la figura
%\includegraphics[width=5cm]{figura.eps}%inserisce una figura larga 5cm
                                        %se si vuole usare va scommentata
%
%%%%%%%%%%%%%%%%%%%%%%%%%%%%%%%%%%%%%%%%%inserisce la legenda ed etichetta
                                        %   la figura con \label{fig:prima}
%\caption[legenda elenco figure]{legenda sotto la figura}\label{fig:prima}
%\end{center}
%\end{figure}

%\section{Seconda Sezione}
%Ora vediamo un elenco puntato:
%\begin{itemize}                         %crea un elenco puntato
%\item primo oggetto
%\item secondo oggetto
%\end{itemize}

%\section{Altra Sezione}
%Vediamo un elenco descrittivo:
%\begin{description}                     %crea un elenco descrittivo
 % \item[OGGETTO1] prima descrizione;
 % \item[OGGETTO2] seconda descrizione;
 % \item[OGGETTO3] terza descrizione.
%\end{description}
%%%%%%%%%%%%%%%%%%%%%%%%%%%%%%%%%%%%%%%%%crea una sottosezione
%\subsection{Altra SottoSezione}
%%%%%%%%%%%%%%%%%%%%%%%%%%%%%%%%%%%%%%%%%crea una sottosottosezione
%\subsubsection{SottoSottoSezione}Questa sottosottosezione non viene
%numerata, ma \`e solo scritta in grassetto.
%\section{Altra Sezione}                 %crea una sottosezione
%Vediamo la creazione di una tabella; la tabella \ref{tab:uno}
%(richiamo il nome della tabella utilizzando la label che ho messo sotto):
%la facciamo di tre righe e tre colonne, la prima colonna
%``incolonnata'' a destra (r) e le altre centrate (c):\\
%\begin{table}[h]                        %ambiente tabella
                                        %(serve per avere la legenda)
%\begin{center}                          %centra nella pagina la tabella
%\begin{tabular}{r|c|c}                  %tre colonne con righe verticali
                                        %   prodotte con |
%\hline \hline                           %inserisce due righe orizzontali
%$(1,1)$ & $(1,2)$ & $(1,3)$\\           %& separa le colonne e con
%\hline                                  %inserisce una riga orizzontale
%$(2,1)$ & $(2,2)$ & $(2,3)$\\           %  \\ va a capo
%\hline                                  %inserisce una riga orizzontale
%$(3,1)$ & $(3,2)$ & $(3,3)$\\
%\hline \hline                           %inserisce due righe orizzontali
%\end{tabular}
%\caption[legenda elenco tabelle]{legenda tabella}\label{tab:uno}
%\end{center}
%\end{table}
%\section{Altra Sezione}\label{sec:prova}%posso mettere le label anche
                                        %   alle section
%\subsection{Listati dei programmi}
%\subsubsection{Primo Listato}
%\begin{verbatim}
%        In questo ambiente     posso scrivere      come voglio,
%lasciare gli spazi che voglio e non % commentare quando voglio
%e ci sar� scritto tutto.
%Quando lo uso � meglio che disattivi il Wrap del WinEdt
%\end{verbatim}
%%%%%%%%%%%%%%%%%%%%%%%%%%%%%%%%%%%%%%%%%non numera l'ultima pagina sinistra
%\clearpage{\pagestyle{empty}\cleardoublepage}
%%%%%%%%%%%%%%%%%%%%%%%%%%%%%%%%%%%%%%%%%per fare le conclusioni
%\chapter*{Conclusioni}
%%%%%%%%%%%%%%%%%%%%%%%%%%%%%%%%%%%%%%%%%imposta l'intestazione di pagina
%\rhead[\fancyplain{}{\bfseries
%CONCLUSIONI}]{\fancyplain{}{\bfseries\thepage}}
%\lhead[\fancyplain{}{\bfseries\thepage}]{\fancyplain{}{\bfseries
%CONCLUSIONI}}
%%%%%%%%%%%%%%%%%%%%%%%%%%%%%%%%%%%%%%%%%aggiunge la voce Conclusioni
                                        %   nell'indice
%\addcontentsline{toc}{chapter}{Conclusioni} Queste sono le
%conclusioni.\\
%In queste conclusioni voglio fare un riferimento alla
%bibliografia: questo \`e il mio riferimento \cite{K3,K4}.
%%%%%%%%%%%%%%%%%%%%%%%%%%%%%%%%%%%%%%%%%imposta l'intestazione di pagina
%\renewcommand{\chaptermark}[1]{\markright{\thechapter \ #1}{}}
%\lhead[\fancyplain{}{\bfseries\thepage}]{\fancyplain{}{\bfseries\rightmark}}
%\appendix                               %imposta le appendici
%\chapter{Prima Appendice}               %crea l'appendice
%In questa Appendice non si \`e utilizzato il comando:\\
%%%%%%%%%%%%%%%%%%%%%%%%%%%%%%%%%%%%%%%%%\verb"" � equivalente all'
                                        %   ambiente verbatim,
                                        %   ma si utilizza all'interno
                                        %   di un discorso.
%\verb"\clearpage{\pagestyle{empty}\cleardoublepage}", ed infatti
%l'ultima pagina 8 ha l'intestazione con il numero di pagina in
%alto.
%%%%%%%%%%%%%%%%%%%%%%%%%%%%%%%%%%%%%%%%%imposta l'intestazione di pagina
%\rhead[\fancyplain{}{\bfseries \thechapter \:Prima Appendice}]
%{\fancyplain{}{\bfseries\thepage}}
%\chapter{Seconda Appendice}             %crea l'appendice
%%%%%%%%%%%%%%%%%%%%%%%%%%%%%%%%%%%%%%%%%imposta l'intestazione di pagina
%\rhead[\fancyplain{}{\bfseries \thechapter \:Seconda Appendice}]
%{\fancyplain{}{\bfseries\thepage}}
%\begin{thebibliography}{90}             %crea l'ambiente bibliografia
%\rhead[\fancyplain{}{\bfseries \leftmark}]{\fancyplain{}{\bfseries
%\thepage}}
%%%%%%%%%%%%%%%%%%%%%%%%%%%%%%%%%%%%%%%%%aggiunge la voce Bibliografia
                                        %   nell'indice
%\addcontentsline{toc}{chapter}{Bibliografia}
%%%%%%%%%%%%%%%%%%%%%%%%%%%%%%%%%%%%%%%%%provare anche questo comando:
%%%%%%%%%%%\addcontentsline{toc}{chapter}{\numberline{}{Bibliografia}}
%\bibitem{K1} Primo oggetto bibliografia.
%\bibitem{K2} Secondo oggetto bibliografia.
%\bibitem{K3} Terzo oggetto bibliografia.
%\bibitem{K4} Quarto oggetto bibliografia.
%\end{thebibliography}
%%%%%%%%%%%%%%%%%%%%%%%%%%%%%%%%%%%%%%%%%non numera l'ultima pagina sinistra
%\clearpage{\pagestyle{empty}\cleardoublepage}
%\chapter*{Ringraziamenti}
%\thispagestyle{empty}
%Qui possiamo ringraziare il mondo intero!!!!!!!!!!\\
%Ovviamente solo se uno vuole, non \`e obbligatorio.
\end{document}
