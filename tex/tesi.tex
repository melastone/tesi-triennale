% !TeX spellcheck = it                                        

\documentclass[12pt,a4paper,openright,twoside]{report}
%openright: apre i capitoli a destra
%twoside: serve per fare un documento fronteretro


\usepackage[italian]{babel}
%\usepackage[latin1]{inputenc}
\usepackage[utf8]{inputenc}
\usepackage[T1]{fontenc} %aumenta il tempo di compilazione


%libreria per impostare il documento (headers & footers)
\usepackage{fancyhdr}


%libreria per avere l'indentazione all'inizio dei capitoli
\usepackage{indentfirst}


%%%%%%%%%libreria per mostrare le etichette
%\usepackage{showkeys}


%%%%%%%%%%%%%%%%%%%%%%%%%%%%%%%%%%%%%%%%%libreria per inserire grafici
%\usepackage{graphicx}


%%%%%%%%%%%%%%%%%%%%%%%%%%%%%%%%%%%%%%%%%libreria per utilizzare font particolari ad esempio  \textsc{}
%\usepackage{newlfont}


%%%%%%%%%%%%%%%%%%%%%%%%%%%%%%%%%%%%%%%%%librerie matematiche
%\usepackage{amssymb}
%\usepackage{amsmath}
%\usepackage{latexsym}
%\usepackage{amsthm}


\oddsidemargin=30pt \evensidemargin=20pt%impostano i margini
\hyphenation{sil-la-ba-zio-ne pa-ren-te-si}%serve per la sillabazione: tra parentesi vanno inserite come nell'esempio le parole che latex non riesce a tagliare nel modo giusto andando a capo.


%%%%%%%%%%%%%%%%%%%%%%%%%%%%%%%%%%%%%%%%%comandi per l'impostazione della pagina, vedi il manuale della libreria fancyhdr per ulteriori delucidazioni
\pagestyle{fancy}\addtolength{\headwidth}{60pt}
\setlength{\headheight}{52pt}
\renewcommand{\chaptermark}[1]{\markboth{\thechapter.\ #1}{}}
\renewcommand{\sectionmark}[1]{\markright{\thesection \ #1}{}}
\rhead[\fancyplain{}{\bfseries\leftmark}]{\fancyplain{}{\bfseries\thepage}}
\cfoot{}


\linespread{1.3}    %imposta l'interlinea


%%%%%%%%%%%%%%%%%%%%%%%%%%%%%%%%%%%%%%%%%definisce nuovi comandi




\textwidth=450pt\oddsidemargin=0pt
\begin{document}

%scrivo il frontespizio
\begin{titlepage}
\begin{center}

{{\Large{\textsc{Alma Mater Studiorum $\cdot$ Universit\`a di
Bologna }}}}

%inserisce le 2 righe orizzontali
\rule[0.1cm]{15.8cm}{0.1mm}
\rule[0.5cm]{15.8cm}{0.6mm}

{\small{\bf SCUOLA DI SCIENZE\\
Corso di Laurea in Informatica }}
\end{center}
\vspace{15mm}
\begin{center}
{\LARGE{\bf Stimare i consumi di gas}}\\
\vspace{3mm}
{\LARGE{\bf nell'esecuzione di Smart Contract}}\\
\vspace{3mm}
{\LARGE{\bf in Ethereum}}\\
\end{center}
\vspace{40mm}

\par
\noindent
\begin{minipage}[t]{0.47\textwidth}
{\large{\bf Relatore:\\
Chiar.mo Prof.\\
Ugo Dal Lago}}
\end{minipage}
\hfill
\begin{minipage}[t]{0.47\textwidth}\raggedleft
{\large{\bf Presentata da:\\
Melania Ghelli}}
\end{minipage}
\vspace{20mm}
\begin{center}
{\large{\bf Sessione II\\%inserire il numero della sessione in cui ci si laurea
Anno Accademico 2018-2019 }}%inserire l'anno accademico a cui si � iscritti
\end{center}
\clearpage{\pagestyle{empty}\cleardoublepage}%non numera l'ultima pagina sinistra
\end{titlepage}

%scrivo la dedica
\begin{titlepage}   %ambiente libero da vincoli


\thispagestyle{empty}   %elimina il numero della pagina
\topmargin=6.5cm 
\raggedleft %incolonna la scrittura a destra
\large  %aumenta la grandezza del carattere a 14pt
\em %emfatizza (corsivo) il carattere

Questa \`e la \textsc{Dedica}:\\
ognuno pu\`o scrivere quello che vuole, \\
anche nulla \ldots  %\ldots lascia tre puntini

\newpage

%%%%%%%%%%%%%%%%%%%%%%%%%%%%%%%%%%%%%%%%
\clearpage{\pagestyle{empty}\cleardoublepage}
\end{titlepage}

\pagenumbering{roman}                   %serve per mettere i numeri romani

\tableofcontents                        %crea l'indice

%%%%%%%%%%%%%%%%%%%%%%%%%%%%%%%%%%%%%%%%%imposta l'intestazione di pagina
\rhead[\fancyplain{}{\bfseries\leftmark}]{\fancyplain{}{\bfseries\thepage}}
\lhead[\fancyplain{}{\bfseries\thepage}]{\fancyplain{}{\bfseries
INDICE}}

%%%%%%%%%%%%%%%%%%%%%%%%%%%%%%%%%%%%%%%%%non numera l'ultima pagina sinistra
\clearpage{\pagestyle{empty}\cleardoublepage}


\chapter*{Introduzione}            
\pagenumbering{arabic}

%%%%%%%%%%%%%%%%%%%%%%%%%%%%%%%%%%%%%%%%%imposta l'intestazione di pagina
\rhead[\fancyplain{}{\bfseries
INTRODUZIONE}]{\fancyplain{}{\bfseries\thepage}}
\lhead[\fancyplain{}{\bfseries\thepage}]{\fancyplain{}{\bfseries
INTRODUZIONE}}

%%%%%%%%%%%%%%%%%%%%%%%%%%%%%%%%%%%%%%%%%aggiunge la voce        Introduzione nell'indice
\addcontentsline{toc}{chapter}{Introduzione}


Le innovazioni tecnologiche introdotte negli ultimi decenni hanno rivoluzionato la nostra società. Il risultato che ne deriva è che numerosi settori stanno cambiando, muovendosi verso una realtà sempre più digitale. Se da una parte questo ha costituito un progresso, dall'altra ha creato nuove opportunità per i cybercriminali.\newline
Oggi il principale obiettivo della sicurezza informatica è proprio quello di trovare soluzioni adattabili alle nuove infrastrutture, come i sistemi IoT che si stanno diffondendo sempre di più. Con l'impiego di queste nuove tecnologie nell'industria i punti di accesso alla rete aziendale sono aumentati, moltiplicando la tipologia ed il numero di minacce. In questo contesto il rischio che si corre è maggiore, poichè legato alla violazione di dati sensibili o addirittura alla compromissione dei processi di produzione. %In questo scenario la cybersecurity è diventata una %disciplina di fondamentale importanza.
\newline
La blockchain nasce in questo contesto, riscuotendo un grande successo grazie al potenziale innovativo che porta con sé. Questa nuova tecnologia permette l'esecuzione di programmi in modo distribuito e sicuro, senza la necessitá di un ente centrale che faccia da garante. Il paradigma
trova applicazioni nei settori piú disparati, offrendo innovazione grazie alla
possibilitá di fare a meno di banche o istituzioni pubbliche.\newline
Ma in cosa consiste dal punto di vista informatico?
Blockchain, catena di blocchi.
Registro distibuito, organizzato in blocchi legati tra loro. Ad interagire con essa sono i miner, coloro che fisicamente realizzano le così dette transazioni. 
Qual è il ruolo del miner nello specifico??\newline
\newline
Grazie alla crittografia (sicurezza informatica comprende una serie di cose, tra cui la crittografia) è stato possibile implementare delle monete virtuali. Da questi studi si è arrivati poi a pensare al bitcoin, che è stato il primo esperimento condotto con successo.\newline
Ciò che indeboliva le crittovalute era il fatto di essere ''centralizzate'', cioè di passare per un ente che ne garantisse l'uso (una sorta di banca).\newline
La blockchain nasce come necessità di superare quest'ostacolo, perché finalmente introduce la possibilità di farne a meno.
Il primo esempio storico di moneta digitale si ha con il Bitcoin. Già da lui si parla di assenza di autorità centrale e rete distibuita - usando algoritmi proof-of-work. Questi permettono di raggiungere un consenso distribuito attraverso tutta la rete. E' il concetto chiave x capire come riusciamo ad evitare di affidarci ad un singolo ente es. Banca.\newline
Tra i sistemi nati grazie alla blockchain troviamo Ethereum, una piattaforma che mette a disposizione un linguaggio di programmazione di alto livello. Questo linguaggio puó essere utilizzato dagli utenti per implementare dei programmi, i così detti smart contract.\newline
In che cosa differisce dal Bitcoin?\newline
\newline
Gli smart contract possono essere eseguiti sulla rete di Ethereum solo al fronte di un pagamento anticipato. Per ragioni di sicurezza a ciascuna istruzione di basso livello è associato un costo monetario. Dunque eseguire un programma costerá tanto quante sono le istruzioni che lo compongono.
Il costo di ciascuna istruzione è espresso in termini di gas, una sorta di carburante che viene pagato in ether, la crittovaluta di Ethereum.
Dal momento che il gas viene pagato in anticipo, potrebbe accadere che l'esecuzione di un programma ecceda la quantitá messa a disposizione. In questi casi la computazione non giunge al termine, risultando nella perdita delle risorse investite dall'utente. Oltre a questo comportamento indesiderato l'esaurimento del gas disponibile puó avere conseguenze pericolose.
Un programma che non gestisce correttamente queste situazioni viene etichettato come vulnerabile. La conseguenza piú diretta è il blocco del contratto, che puó essere anche permanente. La pericolositá peró risiede nel fatto che questo tipo di programmi diventano un bersaglio facile per attacchi malevoli. Vedremo come queste vulnerabilitá possono essere sfruttate per ottenere
comportamenti dannosi per la rete.\newline
Dato il valore monetario associato agli smart contract il rischio che si corre in caso di attacchi informatici è una perdita di denaro. Per questo motivo è necessario individuare possibili criticitá nel codice prima della sua esecuzione. In questo contesto l'analisi statica dei programmi costituisce un potente
strumento di prevenzione.\newline
\newline
All'interno di questo elaborato ci concentreremo solo sulle tecniche di analisi dei consumi di gas. Poter conoscere a priori quest'informazione permetterebbe non solo un investimento adeguato da parte degli utenti, ma anche uno strumento di prevenzione da possibili attacchi.\newline
Attualmente non esistono strumenti in grado di calcolare con precisione la quantitá di gas richiesto durante una computazione. Cercheremo di capirne le ragioni ma soprattutto di individuare dei margini di miglioramento.\newline
\newline


%%%%%%%%%%%%%%%%%%%%%%%%%%%%%%%%%%%%%%%%%non numera l'ultima pagina sinistra
\clearpage{\pagestyle{empty}\cleardoublepage}

%%%%%%%%%%%%%%%%%%%%%%%%%%%%%%%%%%%%%%%%%imposta l'intestazione di pagina per i capitoli successivi
\rhead[\fancyplain{}{\leftmark}]{\fancyplain{}{\thepage}}
\lhead[\fancyplain{}{\thepage}]{\fancyplain{}{\rightmark}}

\chapter{Background}


Lo scopo di questo capitolo è quello di fornire una panoramica dei concetti chiave intorno ai quali si sviluppa l'elaborato.

\section{Blockchain}

Il termine Blockchain - in italiano ``catena di blocchi'' - identifica un registro distribuito e sicuro. In questo senso si può pensare alla blockchain come ad una struttura di dati simile ad una lista crescente, dove le informazioni sono raggruppate in blocchi collegati fra loro.\newline
Ciascun blocco codifica una sequenza di transazioni individuale, e viene concatenato a quello precedente seguendo un ordine cronologico. La concatenazione è irreversibile: ciascun nuovo blocco contiene la firma digitale di quello precedente. In questo modo, modificare un blocco implicherebbe l'invalidazione di tutta la catena successiva.\newline  
La peculiarità di questa struttura risiede nel fatto che sia condivisa: ogni nodo che compone la rete mantiene una copia del registro aggiornata. Per poter aggiungere un blocco è dunque necessario validare l'intera catena, ed ottenere un consenso da parte degli altri nodi della rete. Una volta ottenuto, il nuovo blocco viene trasmesso agli altri componenti in modo tale da aggiornare lo stato della blockchain.\newline
Il processo di validazione dei nuovi blocchi viene realizzato dai miner.
Il loro compito è quello di verificare le transazioni proposte e fare in modo che il nuovo blocco venga linkato alla blockchain. Per fare questo i miner sono chiamati a risolvere un algoritmo proof-of-work, un puzzle crittografico che richiede un significativo costo computazionale per essere risolto.\newline
Questo sistema permette di raggiungere il consenso senza la necessità di un'autorità centrale che faccia da garante. \'E il concetto chiave delle tecnologie basate su blockchain: la possibilità di implementare servizi sicuri senza appoggiarsi a banche, istituzioni pubbliche, ecc.\newline
\newline
Questa nuova tecnologia può essere integrata in diverse aree \cite{K1}, sebbene ad oggi il suo uso più conosciuto sia quello nei sistemi di pagamento che impiegano crittovalute.
Il dato non è poi così sorprendente: la prima blockchain nasce grazie a Satoshi Nakamoto assieme al Bitcoin \cite{K2}. In questo senso il Bitcoin è una
piattaforma di pagamenti, dove la catena di blocchi funge da storico di tutte le transazioni avvenute: una sorta di conto corrente condiviso.\newline

\section{Ethereum, gli Smart Contract e la EVM}

All'interno di quest'elaborato verrà presa in considerazione solo il network Ethereum, una piattaforma decentralizzata basata su una blockchain, che come Bitcoin possiede una propria valuta: l'\textit{ether}.\newline 
Diversamente da quanto vale per le altre crittovalute, Ethereum non è solo un network per lo scambio di moneta, ma un framework che permette l'esecuzione di programmi. Tali programmi prendono il nome di \textit{smart contract}, cioè ``contratti intelligenti''. Sebbene il nome possa suggerire una funzione ben precisa, questi programmi sono usati per computazioni general-purpose, permettendo quindi di realizzare un vasto numero di operazioni.\newline
Gli smart contract sono scritti in linguaggi ad alto livello; fra i vari (Serpent, Viper e LLL) quello più diffuso ad oggi è Solidity \cite{K3}. 
Tale linguaggio object-oriented è pensato solo per lo sviluppo di smart contract che, per poter girare nella rete, vengono poi tradotti in bytecode. Ciascun nodo di Ethereum infatti esegue localmente la Ethereum Virtual Machine, anche detta EVM, una macchina a stack in grado di eseguire un linguaggio di basso livello, ossia bytecode. Questo linguaggio è non tipato, e composto da un piccolo insieme di istruzioni.\newline 

\section{Il ruolo del gas}

Per \textit{gas} si intende l'unità di misura dello sforzo computazionale richiesto dalla EVM per eseguire ciascuna istruzione. Diremo quindi che eseguire uno smart contract costa una certa quantità di gas.\newline
Nello specifico ciascuna istruzione di basso livello ha associato un costo fisso in gas. Per calcolare quindi il consumo totale di un programma Solidity è necessario comprendere in quali istruzioni di basso livello verrà tradotto.\newline
I costi di alcune delle istruzioni EVM \cite{K4} sono riportati nella tabella \ref{tab:gas-costs}.
%(richiamo il nome della tabella utilizzando la label che ho messo sotto)

\begin{table}[h]                        %ambiente tabella
                                        %(serve per avere la legenda)
\begin{center}  %centra nella pagina la tabella

\begin{tabular}{p{5cm}rp{6cm}}  

\hline \hline   %inserisce due righe orizzontali
Istruzione & Costo & Descrizione\\   %& separa le colonne
\hline  %inserisce una riga orizzontale
\bf JUMPDEST & 1 & Indica la destinazione di un'istruzione JUMP\\
\bf POP & 2 & Rimuove un elemento dallo stack\\
\bf PUSHn & 3 & Inserisce un elemento di n byte nello stack\\
\bf ADD, SUB & 3 & Operatori aritmetici\\
\bf AND, OR, NOT, XOR, ISZERO, BYTE & 3 & Operatori logici\\
\bf MUL, DIV & 5 & Operatori aritmetici\\
\bf JUMP & 8 & Salto semplice senza condizione\\
\bf JUMP1 & 10 & Salto condizionale\\
\hline
\bf CALL & 700 & Chiama una transazione\\
\bf CALLVALUE & 9000 & Pagato per un argomento diverso da 0 dell'istruzione CALL\\
\bf SSTORE & 20000 & Salva una parola in memoria. Si paga quando il valore precedente è uguale a 0\\
\hline \hline
\end{tabular}

\caption[legenda elenco tabelle]{Costi espressi in gas di alcune istruzioni della EVM}\label{tab:gas-costs}
\end{center}
\end{table}

Il gas viene pagato in ether dagli utenti che intendono far eseguire una \textit{transazione}. Con transazione si intende l'azione di creare uno smart contract o di chiamarne delle funzioni e di pagare un miner per far eseguire la transazione stessa.\newline
Per fare in modo che la sua transazione venga scelta, un utente stabilisce la quantità di gas che è disposto a pagare per farla portare a termine. Il miner poi, in base a questo parametro, sceglie quali transazioni eseguire. \'E importante fornire un'adeguata somma: i miner, al termine della transazione, possono beneficiare dell'ether destinato all'acquisto di gas che è rimasto inutilizzato. Questo significa che gli smart contract con più probabilità di essere scelti sono quelli degli utenti disposti a pagare di più.\newline
La scelta adottata da Ethereum di far pagare i propri utenti non è triviale. Prima di tutto impedisce loro di sovraccaricare i miner di lavoro sfruttando il potere computazionale della rete. Inoltre scoraggia gli utenti a impiegare troppa memoria, una risorsa preziosa nelle tecnologie basate su blockchain. Infine limita il numero di computazioni eseguite dalla stessa transazione. Questo difende il network intero da attacchi malevoli come i DDoS: la finitezza delle transazioni fa sì che non si possa, ad esempio, far ciclare un programma infinitamente. Per poter disabilitare la rete anche solo per pochi minuti gli hacker dovrebbero pagare delle ingenti somme.\newline
Dal momento che il gas viene pagato anticipatamente, può succedere che durante la sua esecuzione un programma ecceda la quantità che ha a disposizione. Questo comportamento è  indesiderato, in quanto comporta spiacevoli conseguenze. La più immediata è il blocco della transazione: la computazione non giunge a termine, l'utente non ottiene il risultato desiderato e l'ether pagato per il gas va perso.\newline 
La conseguenza indiretta invece è il possibile blocco permanente dello smart contract. Quando durante l'esecuzione di una transazione un'istruzione richiede una quantità di gas superiore a quella disponibile, la EVM solleva un'eccezione di tipo \textit{out-of-gas} e interrompe la transazione. Qualora lo smart contract non preveda una gestione di questo tipo di eccezione resterà bloccato per sempre.\newline






%%%%%%%%%%%%%%%%%%%%%%%%%%%%%%%%%%%%%%%%%non numera l'ultima pagina sinistra
\clearpage{\pagestyle{empty}\cleardoublepage}

\chapter{Analisi statica}

\section{Tecniche in informatica}

\section{Analisi statica di smart contract}

\section{Tool per l'analisi}

%\listoffigures                          %crea l'elenco delle figure

%\listoftables                           %crea l'elenco delle tabelle
%%%%%%%%%%%%%%%%%%%%%%%%%%%%%%%%%%%%%%%%%non numera l'ultima pagina sinistra
%\clearpage{\pagestyle{empty}\cleardoublepage}


%\begin{figure}[h]                       %crea l'ambiente figura; [h] sta
                                        %   per here, cio� la figura va qui
%\begin{center}                          %centra nel mezzo della pagina
                                        %   la figura
%\includegraphics[width=5cm]{figura.eps}%inserisce una figura larga 5cm
                                        %se si vuole usare va scommentata
%
%%%%%%%%%%%%%%%%%%%%%%%%%%%%%%%%%%%%%%%%%inserisce la legenda ed etichetta
                                        %   la figura con \label{fig:prima}
%\caption[legenda elenco figure]{legenda sotto la figura}\label{fig:prima}
%\end{center}
%\end{figure}


%%%%%%%%%%%%%%%%%%%%%%%%%%%%%%%%%%%%%%%%%%ELENCHI

%Ora vediamo un elenco numerato:         %crea un elenco numerato
%\begin{enumerate}
%\item primo oggetto
%\item secondo oggetto
%\item terzo oggetto
%\item quarto oggetto
%\end{enumerate}

%Ora vediamo un elenco puntato:
%\begin{itemize}                         %crea un elenco puntato
%\item primo oggetto
%\item secondo oggetto
%\end{itemize}


%Vediamo un elenco descrittivo:
%\begin{description}                     %crea un elenco descrittivo
 % \item[OGGETTO1] prima descrizione;
 % \item[OGGETTO2] seconda descrizione;
 % \item[OGGETTO3] terza descrizione.
%\end{description}

%%%%%%%%%%%%%%%%%%%%%%%%%%%%%%%%%%%%%%%%%crea una sottosezione
%\subsection{Altra SottoSezione}
%%%%%%%%%%%%%%%%%%%%%%%%%%%%%%%%%%%%%%%%%crea una sottosottosezione
%\subsubsection{SottoSottoSezione}Questa sottosottosezione non viene
%numerata, ma \`e solo scritta in grassetto.

%\section{Altra Sezione}                 %crea una sottosezione

%\section{Altra Sezione}\label{sec:prova}%posso mettere le label anche
                                        %   alle section
%\subsection{Listati dei programmi}
%\subsubsection{Primo Listato}
%\begin{verbatim}
%        In questo ambiente     posso scrivere      come voglio,
%lasciare gli spazi che voglio e non % commentare quando voglio
%e ci sar� scritto tutto.
%Quando lo uso � meglio che disattivi il Wrap del WinEdt
%\end{verbatim}


%%%%%%%%%%%%%%%%%%%%%%%%%%%%%%%%%%%%%%%%%non numera l'ultima pagina sinistra
\clearpage{\pagestyle{empty}\cleardoublepage}


%%%%%%%%%%%%%%%%%%%%%%%%%%%%%%%%%%%%%%%%%per fare le conclusioni
\chapter*{Conclusioni}

%%%%%%%%%%%%%%%%%%%%%%%%%%%%%%%%%%%%%%%%%imposta l'intestazione di pagina
\rhead[\fancyplain{}{\bfseries
CONCLUSIONI}]{\fancyplain{}{\bfseries\thepage}}
\lhead[\fancyplain{}{\bfseries\thepage}]{\fancyplain{}{\bfseries
CONCLUSIONI}}

%%%%%%%%%%%%%%%%%%%%%%%%%%%%%%%%%%%%%%%%%aggiunge la voce Conclusioni
                                        %   nell'indice
\addcontentsline{toc}{chapter}{Conclusioni} 


%%%%%%%%%%%%%%%%%%%%%%%%%%%%%%%%%%%%%%%%%imposta l'intestazione di pagina
\renewcommand{\chaptermark}[1]{\markright{\thechapter \ #1}{}}
\lhead[\fancyplain{}{\bfseries\thepage}]{\fancyplain{}{\bfseries\rightmark}}


%\appendix                               %imposta le appendici

%\chapter{Prima Appendice}               %crea l'appendice
%In questa Appendice non si \`e utilizzato il comando:\\
%%%%%%%%%%%%%%%%%%%%%%%%%%%%%%%%%%%%%%%%%\verb"" � equivalente all'
                                        %   ambiente verbatim,
                                        %   ma si utilizza all'interno
                                        %   di un discorso.
%\verb"\clearpage{\pagestyle{empty}\cleardoublepage}", ed infatti
%l'ultima pagina 8 ha l'intestazione con il numero di pagina in
%alto.

%%%%%%%%%%%%%%%%%%%%%%%%%%%%%%%%%%%%%%%%%imposta l'intestazione di pagina
%\rhead[\fancyplain{}{\bfseries \thechapter \:Prima Appendice}]
%{\fancyplain{}{\bfseries\thepage}}


%\chapter{Seconda Appendice}             %crea l'appendice

%%%%%%%%%%%%%%%%%%%%%%%%%%%%%%%%%%%%%%%%%imposta l'intestazione di pagina
%\rhead[\fancyplain{}{\bfseries \thechapter \:Seconda Appendice}]
%{\fancyplain{}{\bfseries\thepage}}

\begin{thebibliography}{90}             %crea l'ambiente bibliografia

%\rhead[\fancyplain{}{\bfseries \leftmark}]{\fancyplain{}{\bfseries
%\thepage}}

%%%%%%%%%%%%%%%%%%%%%%%%%%%%%%%%%%%%%%%%%aggiunge la voce Bibliografia
                                        %   nell'indice
%\addcontentsline{toc}{chapter}{Bibliografia}

%%%%%%%%%%%%%%%%%%%%%%%%%%%%%%%%%%%%%%%%%provare anche questo comando:
%%%%%%%%%%%\addcontentsline{toc}{chapter}{\numberline{}{Bibliografia}}
\bibitem{K1} M. Iansiti e K. R. Lakhani. `` The Truth About Blockchain'' hbr.org.\\https://hbr.org/2017/01/the-truth-about-blockchain (Ultimo accesso 24 novembre 2019) 
\bibitem{K2} S. Nakamoto. ``Bitcoin: A peer-to-peer electronic cash system''
(Bitcoin whitepaper). 31 Ottobre 2008. https://bitcoin.org/bitcoin.pdf
\bibitem{K3} Ethereum Foundation. ``The solidity contract-oriented programming language''. https://github.com/ethereum/solidity
\bibitem{K4} Dr. Gavin Wood. ``Ethereum:
A secure decentralised generalised transaction ledger''. Versione: 20 Ottobre 2019. https://ethereum.github.io/yellowpaper/paper.pdf
\end{thebibliography}

%%%%%%%%%%%%%%%%%%%%%%%%%%%%%%%%%%%%%%%%%non numera l'ultima pagina sinistra
\clearpage{\pagestyle{empty}\cleardoublepage}

%\chapter*{Ringraziamenti}
%\thispagestyle{empty}
%Qui possiamo ringraziare il mondo intero!!!!!!!!!!\\
%Ovviamente solo se uno vuole, non \`e obbligatorio.

\end{document}
