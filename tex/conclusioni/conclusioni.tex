L'utilizzo dell'analisi statica nella verifica delle proprietà di sicurezza dei contratti di Ethereum ha riscosso successo recentemente, portando allo sviluppo di alcuni software che combinano diverse tecniche di analisi statica. Solo una piccola porzione di questi programmi si focalizza sui consumi di gas; abbiamo visto \textsc{MadMax}~\cite{grech2018madmax}, che conduce un'analisi orientata a individuare vulnerabilità nel codice legate al gas. Un altro tool simile è \textsc{Gasper}~\cite{chen2017under}, che grazie all'analisi dei consumi è in grado di individuare dei pattern ai quali è associato un elevato consumo di gas; offre dunque un servizio di ottimizzazione del codice. Nessuno di questi però è in grado di produrre dei bound.\newline
\indent La stima dei consumi di gas resta quindi un topic poco trattato. I lavori condotti in questa direzione sono ancora pochi: oltre al tool \textsc{Gastap}~\cite{DBLP:journals/corr/abs-1811-10403}, sul quale ci siamo focalizzati durante questa trattazione, è importante citare anche il lavoro di Marescotti et al.~\cite{marescotti2018computing}, che sottolineando l'importanza del tema, propone due algoritmi per la stima dei consumi di gas; questo approccio tuttavia non è stato ancora implementato, motivo per cui non è stato possibile includerlo in questo lavoro.\newline
\indent Va inoltre considerato che questi strumenti hanno ancora molte limitazioni, impedendo la verifica di programmi sofisticati. Costrutti come i cili while o la ricorsione non vengono gestiti correttamente, ponendo una forte limitazione allo sviluppatore che desideri verificare i consumi del proprio programma. \`E questo uno dei motivi per cui l'analisi statica non è ancora molto utilizzata nella stima del gas, la ricerca dovrebbe quindi cercare di studiare i limiti di questo problema, assieme a tecniche di analisi statica più espressive.\newline
