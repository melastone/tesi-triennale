\documentclass[a4paper,12pt]{report}
\usepackage[italian]{babel}
\usepackage[latin1]{inputenc}


%opening
\title{Stimare i consumi di gas nell'esecuzione di Smart Contract in Ethereum}
\author{Melania Ghelli}

\begin{document}

\maketitle
\chapter*{Introduzione}


Negli ultimi anni la blockchain ha riscosso molto successo. Questa nuova tecnologia permette l'esecuzione di programmi in modo distribuito e sicuro, senza la necessit\'a di un ente centrale che faccia da garante. Il paradigma trova applicazioni nei settori pi\'u disparati, offrendo innovazione grazie alla possibilit\'a di fare a meno di banche o istituzioni pubbliche.\newline
Tra i sistemi nati grazie a questa tecnologia troviamo Ethereum, una piattaforma che mette a disposizione un linguaggio di programmazione di alto livello. Questo linguaggio pu\'o essere utilizzato dagli utenti per implementare dei programmi, i cos\'i detti smart contract.\newline
Gli smart contract possono essere eseguiti sulla rete di Ethereum solo al fronte di un pagamento anticipato. Per ragioni di sicurezza a ciascuna istruzione di basso livello \'e associato un costo monetario. Dunque eseguire un programma coster\'a tanto quante sono le istruzioni che lo compongono.\newline
Il costo di ciascuna istruzione \'e espresso in termini di gas, una sorta di carburante che viene pagato in ether, la crittovaluta di Ethereum.\newline
Dal momento che il gas viene pagato in anticipo, potrebbe accadere che l'esecuzione di un programma ecceda la quantit\'a messa a disposizione. In questi casi la computazione non giunge al termine, risultando nella perdita delle risorse investite dall'utente.
Oltre a questo comportamento indesiderato l'esaurimento del gas disponibile pu\'o avere conseguenze pericolose.\newline
Un programma che non gestisce correttamente queste situazioni viene etichettato come vulnerabile. La conseguenza pi\'u diretta \'e il blocco del contratto, che pu\'o essere anche permanente. La pericolosit\'a per\'o risiede nel fatto che questo tipo di programmi diventano un bersaglio facile per  attacchi malevoli. Vedremo come queste vulnerabilit\'a possono essere sfruttate per ottenere comportamenti dannosi per la rete.\newline
Dato il valore monetario associato agli smart contract il rischio che si corre in caso di attacchi informatici \'e una perdita di denaro. Per questo motivo \'e necessario individuare possibili criticit\'a nel codice prima della sua esecuzione. In questo contesto l'analisi statica dei programmi costituisce un potente strumento di prevenzione.\newline
All'interno di questo elaborato ci concentreremo solo sulle tecniche di analisi dei consumi di gas. Poter conoscere a priori quest'informazione permetterebbe non solo un investimento adeguato da parte degli utenti, ma anche uno strumento di prevenzione da possibili attacchi.\newline
Attualmente non esistono strumenti in grado di calcolare con precisione la quantit\'a di gas richiesto durante una computazione. Cercheremo di capirne le ragioni ma soprattutto di individuare dei margini di miglioramento.  


\end{document}
