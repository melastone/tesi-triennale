
% Mi piacerebbe nella parte 1 elencare delle challenge di cybersecurity a cui potermi riallacciare in fondo alla parte 2 proponendo la blockchain come possibile soluzione

%1
Le innovazioni tecnologiche introdotte negli ultimi decenni hanno rivoluzionato la nostra società. Il risultato che ne deriva è che numerosi settori stanno cambiando, muovendosi verso una realtà sempre più digitale. Se da una parte questo ha costituito un progresso, dall'altra ha creato nuove opportunità per i cybercriminali.\newline
\indent Oggi il principale obiettivo della sicurezza informatica è proprio quello di trovare soluzioni adattabili alle nuove infrastrutture, come i sistemi IoT ~\cite{KHAN2018395} che si stanno diffondendo in modo capillare. Con l'impiego di queste nuove tecnologie nell'industria i punti di accesso alla rete aziendale sono aumentati, moltiplicando la tipologia ed il numero di minacce. In questo contesto il rischio che si corre è maggiore, poichè legato alla violazione di dati sensibili o addirittura alla compromissione dei processi di produzione.\newline

%2
La blockchain nasce nel 2008 assieme al Bitcoin ~\cite{nakamoto2008bitcoin}, e ad oggi riveste un ruolo chiave nel settore della cybersecurity, offrendo sicurezza, anonimato e integrità dei dati senza la necessità di un ente centrale che faccia da garante ~\cite{dai2017bitcoin}.\newline
\indent Dal punto di vista informatico il termine Blockchain identifica un registro distibuito, organizzato in blocchi legati tra loro. Ciascun blocco raggruppa un numero arbitrario di \textit{transazioni}, vale a dire l'unità di base in cui le informazioni vengono codificate nel registro. Il contenuto di ciascuna transazione varia a seconda del protocollo adottato dalla blockchain. Ad interagire con la blockchain sono i miner, coloro che effettivamente realizzano le transazioni per conto degli utenti della rete. Si occupano di raggruppare le transazioni in blocchi, per poi farli approvare dal resto della rete. Per ottenere il consenso delle altre parti è richiesta la risoluzione di un algoritmo chiamato \textit{proof-of-work}.\newline
\indent Il paradigma trova applicazioni nei settori piú disparati ~\cite{iansiti_lakhani_2017}, offrendo innovazione grazie alla possibilitá di fare a meno di banche o istituzioni pubbliche. Può essere utilizzata come tecnologia di base dalla quale partire per implementare dei servizi efficienti.\newline
%Altri utilizzi..

% Mi riallaccerei qui, parlando delle possibili applicazioni della blockchain

%3
\indent Tuttavia ad oggi le applicazioni più diffuse della blockchain includono l'impiego delle crittovalute. La più celebre è sicuramente il Bitcoin, il primo esempio storico di moneta digitale che introduce i concetti di rete distibuita e algoritmo proof-of-work. Questo protocollo permette di raggiungere un consenso distribuito attraverso tutto il network. \`E il concetto chiave per capire come riusciamo ad evitare di affidarci ad un'autorità centrale.\newline
\indent Tra le principali crittovalute troviamo Ethereum, che con una quotazione attuale di circa 140\euro{} è seconda al Bitcoin. Ethereum si differenzia dalla sua concorrente per i servizi offerti: si tratta di una piattaforma open-source che mette a disposizione un linguaggio di programmazione Turing-completo. Questo linguaggio puó essere utilizzato dagli utenti per implementare dei programmi, i così detti smart contract.\newline

%4
\indent Gli smart contract possono essere eseguiti sul network peer-to-peer di Ethereum solo al fronte di un pagamento anticipato. Per ragioni di sicurezza a ciascuna istruzione di basso livello è associato un costo monetario. Dunque eseguire un programma costerá tanto quante sono le istruzioni che lo compongono. Il costo di ciascuna istruzione è espresso in termini di gas, una sorta di carburante che viene pagato in ether, la crittovaluta di Ethereum. Dal momento che il gas viene pagato in anticipo, potrebbe accadere che l'esecuzione di un programma ecceda la quantitá messa a disposizione. In questi casi la computazione non giunge al termine, risultando nella perdita delle risorse investite dall'utente. Oltre a questo comportamento indesiderato l'esaurimento del gas disponibile puó avere conseguenze pericolose. Un programma che non gestisce correttamente queste situazioni viene etichettato come vulnerabile. La conseguenza piú diretta è il blocco del contratto, che puó essere anche permanente. La pericolositá peró risiede nel fatto che questo tipo di programmi diventano un bersaglio facile per attacchi malevoli. Dato il valore monetario associato agli smart contract il rischio che si corre in caso di attacchi informatici è una ingente perdita di denaro. Un caso del genere si è verificato con l'attacco DAO ~\cite{daian2016dao} che nel 2016 ha comportato la perdita di circa 150 milioni di dollari.\newline 
\indent Per queste ragioni è necessario individuare possibili criticitá nel codice prima della sua esecuzione. In questo contesto l'analisi statica dei programmi costituisce un potente strumento di prevenzione.\newline

%5
\indent All'interno di questo elaborato ci concentreremo solo sulle tecniche di analisi dei consumi di gas. Conoscere a priori quest'informazione permetterebbe non solo un investimento adeguato da parte degli utenti, ma anche uno strumento di difesa da eventuali attacchi. Nonostante i numerosi vantaggi che potrebbe apportare, l'utilizzo dell'analisi statica nello sviluppo degli smart contract è ancora una pratica poco diffusa. Come vedremo, una delle motivazioni più forti è l'assenza di strumenti che permettano di calcolare con precisione la quantitá di gas richiesto durante una computazione. L'obiettivo principale di questa tesi è quello di individuare dei tool capaci di condurre questo tipo di analisi, e conseguentemente verificarne la precisione.\newline
