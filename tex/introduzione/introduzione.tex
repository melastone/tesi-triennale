
Le innovazioni tecnologiche introdotte negli ultimi decenni hanno rivoluzionato la nostra società. Il risultato che ne deriva è che numerosi settori stanno cambiando, muovendosi verso una realtà sempre più digitale. Se da una parte questo ha costituito un progresso, dall'altra ha creato nuove opportunità per i cybercriminali.\newline
Oggi il principale obiettivo della sicurezza informatica è proprio quello di trovare soluzioni adattabili alle nuove infrastrutture, come i sistemi IoT che si stanno diffondendo sempre di più. Con l'impiego di queste nuove tecnologie nell'industria i punti di accesso alla rete aziendale sono aumentati, moltiplicando la tipologia ed il numero di minacce. In questo contesto il rischio che si corre è maggiore, poichè legato alla violazione di dati sensibili o addirittura alla compromissione dei processi di produzione. %In questo scenario la cybersecurity è diventata una %disciplina di fondamentale importanza.
\newline
La blockchain nasce in questo contesto, riscuotendo un grande successo grazie al potenziale innovativo che porta con sé. Questa nuova tecnologia permette l'esecuzione di programmi in modo distribuito e sicuro, senza la necessitá di un ente centrale che faccia da garante. Il paradigma
trova applicazioni nei settori piú disparati, offrendo innovazione grazie alla
possibilitá di fare a meno di banche o istituzioni pubbliche.\newline
Ma in cosa consiste dal punto di vista informatico?
Blockchain, catena di blocchi.
Registro distibuito, organizzato in blocchi legati tra loro. Ad interagire con essa sono i miner, coloro che fisicamente realizzano le così dette transazioni. 
Qual è il ruolo del miner nello specifico??\newline
\newline
Grazie alla crittografia (sicurezza informatica comprende una serie di cose, tra cui la crittografia) è stato possibile implementare delle monete virtuali. Da questi studi si è arrivati poi a pensare al bitcoin, che è stato il primo esperimento condotto con successo.\newline
Ciò che indeboliva le crittovalute era il fatto di essere ''centralizzate'', cioè di passare per un ente che ne garantisse l'uso (una sorta di banca).\newline
La blockchain nasce come necessità di superare quest'ostacolo, perché finalmente introduce la possibilità di farne a meno.
Il primo esempio storico di moneta digitale si ha con il Bitcoin. Già da lui si parla di assenza di autorità centrale e rete distibuita - usando algoritmi proof-of-work. Questi permettono di raggiungere un consenso distribuito attraverso tutta la rete. E' il concetto chiave x capire come riusciamo ad evitare di affidarci ad un singolo ente es. Banca.\newline
Tra i sistemi nati grazie alla blockchain troviamo Ethereum, una piattaforma che mette a disposizione un linguaggio di programmazione di alto livello. Questo linguaggio puó essere utilizzato dagli utenti per implementare dei programmi, i così detti smart contract.\newline
In che cosa differisce dal Bitcoin?\newline
\newline
Gli smart contract possono essere eseguiti sulla rete di Ethereum solo al fronte di un pagamento anticipato. Per ragioni di sicurezza a ciascuna istruzione di basso livello è associato un costo monetario. Dunque eseguire un programma costerá tanto quante sono le istruzioni che lo compongono.
Il costo di ciascuna istruzione è espresso in termini di gas, una sorta di carburante che viene pagato in ether, la crittovaluta di Ethereum.
Dal momento che il gas viene pagato in anticipo, potrebbe accadere che l'esecuzione di un programma ecceda la quantitá messa a disposizione. In questi casi la computazione non giunge al termine, risultando nella perdita delle risorse investite dall'utente. Oltre a questo comportamento indesiderato l'esaurimento del gas disponibile puó avere conseguenze pericolose.
Un programma che non gestisce correttamente queste situazioni viene etichettato come vulnerabile. La conseguenza piú diretta è il blocco del contratto, che puó essere anche permanente. La pericolositá peró risiede nel fatto che questo tipo di programmi diventano un bersaglio facile per attacchi malevoli. Vedremo come queste vulnerabilitá possono essere sfruttate per ottenere
comportamenti dannosi per la rete.\newline
Dato il valore monetario associato agli smart contract il rischio che si corre in caso di attacchi informatici è una perdita di denaro. Per questo motivo è necessario individuare possibili criticitá nel codice prima della sua esecuzione. In questo contesto l'analisi statica dei programmi costituisce un potente
strumento di prevenzione.\newline
\newline
All'interno di questo elaborato ci concentreremo solo sulle tecniche di analisi dei consumi di gas. Poter conoscere a priori quest'informazione permetterebbe non solo un investimento adeguato da parte degli utenti, ma anche uno strumento di prevenzione da possibili attacchi.\newline
Attualmente non esistono strumenti in grado di calcolare con precisione la quantitá di gas richiesto durante una computazione. Cercheremo di capirne le ragioni ma soprattutto di individuare dei margini di miglioramento.\newline
\newline
